\chapter{Development Methodology}
This section contains descriptions regarding the different development
methodolo- gies that have been brought up and that have been researched. Each subsection
includes both a short explanation, advantages and drawbacks for each methodol-
ogy.

\subsection{The Waterfall Method}
Figure x.x shows a graphical explanation of the sequential design process called the waterfall method. This method has been around for decades. The waterfall method is based on the idea of visiting each of the phases, Initiation, Analysis, Design, Construction, Testing, Implementation and Maintenance, only once and finish one before starting the next. The name is given from the idea of progress flowing through each face, like a waterfall. This results in huge challenges regarding controlling dependencies if the project doeas reiteration over previous phases at a later stage.

The main advantage to the waterfall method is that bugs and changes are cheaper to fix if you fix them right away, as it will save you a lot of time/money later on.
The main drawbacks are as mentioned earlier, that once the project has moved on to the next phase and the team should not backtrack and edit the previously completed phases, since this might make the further implementation more difficult. The fact that planning has to be done very thoroughly in the beginning to avoid having to reiterate previous phases at a later stage as this can be costly and completed, is also a disadvantage. Leading to a problem with projects where there is no overview to what is to be done and how long time it will take, this method will lead to uncertainty. A roll back to an earlier stage will most likely prove early estimates wrong and might cause complications to the development.

\subsection{Scrum}
\label{sec:scrum}
Figure x.x shows a graphical explanation of the Scrum method, one of many agile development methods. Agile development meaning that it is an iterative, incremental model which emphasizes on doing several short sprints where the goal is to complete some smaller set of tasks. After a given period, usually one to four weeks, the development team summarizes what have been done and what is left from the current sprint, needing to be completed in the upcoming sprints.
The advantages of scrum are that it makes the software development more versatile, the team can work on all phases and parts of the project at the same time, and update earlier assumptions based on newer discoveries. Meaning that requirements and modelling does need to be finished before starting implementation and because of this, changes are less expensive to do. This is done by having a more relaxed relationship to documentation of source code and the process.
Nothing is written in stone until the product is done, as opposed to the waterfall method, mentioned earlier.
The main drawback of Scrum is the complexity of it. All methods has a certain learning curve at the beginning, leading to stress or less effective work. Scrum has several submethods, all with small differences. This may lead to a learning curve for experienced users.
A more specific explanation of how this project used Scrum is found in Section x.x

\subsection{Choice of methodology}
The development chose the Scrum methodology instead of the Waterfall method, due to many reasons. Foremost, the customer asked the team to work in the Scrum methodology. "We want the process to be as agile as possible, to a certain level. Waterfall will not suffice". The customer had many ideas regarding the layout and the functionality of the applications, and were not sure what to include. This leading to a situation where spending time making a detailed requirement specification and locking down all the details was pointless.
The customer was likely to make changes to the initial requirements once the first plan was ready. Secondly, the team was way more eager to try out scrum than to use waterfall. The simple fact that scrum is highly recommended by real-life developers, is a good argument for doing so. The developers were also eager to learn more about Scrum, as not all had used it before.


\subsection{Sprints}
\label{sec:sprints}
This section gives a short description of how the Scrum development method was used in the project. 
For a general explanation of Scrum see Section \ref{sec:scrum}.

\subsubsection{Sprint duration}
We decided on having 14 day sprints. After discussions with the customer,
we agreed that this would be a suitable duration, due to the fact 
that the documentation needed for each sprint would be time consuming 
for shorter sprints.

\subsubsection{Sprint Planning Meeting}
To start each sprint, we held a sprint planning meeting. During this meeting, we discussed which user 
stories/epics from the sprint backlog should be worked on during the sprint. The reason behind such a 
meeting is to make sure the team is on updated on the goals for the following sprint. To decide what 
user stories/epics should be chosen, the priorities given by the customer was used as a pinpoint. If 
the customer wanted to make any changes during a sprint, the changes were noted and discussed during 
the next sprint planning meeting.

\subsubsection{Daily Standup}
The daily standups (also commonly known as daily scrum meeting) were held on Mondays, Wednesdays and 
Fridays. The team decided on this semi-daily recurrence since not all team members were able to work 
on the project every day. During the standup meetings all team members would answer three questions: 
what have you done since our last meeting, what will you work on until the next meeting, and what 
problems did occur since our last meeting?

Answering these questions gave a certain status update, and made it easier to re-assign team members 
to tasks if needed. During the standups all technical discussions were discouraged. If any technical 
questions arose, the people involved would discuss this after the meeting, to make sure they were not 
wasting other people's time. Each standup had a max allowed length of 15 minutes. 

\subsubsection{Sprint retrospective}
The sprint retrospective is the written conclusion of the sprint. A meeting was held at the end of each 
sprint, discussing the results throughout the sprint, both finished and unfinished tasks. The tasks not 
completed were moved to the next sprint, and the reason for the task not being completed was stated in 
the sprint report. 

The sprint report also includes an update of the sprint backlog, along with an overview of how much time 
was spent on each task, making it easy to compare to the time estimate. 

The sprint retrospective also contains a burndown chart, giving a visual representation of how the team 
worked during the sprint.

For each sprint we answered the following questions: 
\begin{itemize}
	\item What went well?
	\item What shall we start doing?
	\item What could have gone better?
	\item What should we stop doing?
\end{itemize}
%TODO: the questions

\subsubsection{Explanation of Sprint Backlog}
The sprint backlog is a task management tool to document and ensure the progress of the sprint. Each task the 
team chooses to focus on in the sprint is enlisted. The task is given an ID and already has a name. The 
Function number is an hour-independent number telling how difficult the team expects the task to be. The 
base number represents how many hours the team expects to work to finish one story point. The base number multiplied 
with the function number for a task gives the estimated work hours needed to finish a task.

The base number may change from sprint to sprint, but not during a sprint. The team did an evaluation of 
the base number in advance of each sprint, to make good estimates.

The name column is used to keep track of who is responsible for the task. This may change during the sprint, 
but the sprint backlog should always be showing the correct info.

Based on how many team members are available and how many work hours they may put in, the team gives an 
expected decrease of the story points left. This is reflected in the sprint burndown chart for each sprint, 
as a straight decreasing line.
