\section{Karotz}
The Karotz is a robot shaped as a bunny that can interact with
a user through light, ear movement and sound. It can also take
input through a button, moving its ears, an RFID chip, voice
commands and serial (internet) communication.

The project includes developing an application for the Karotz
platform that will serve as an addition to the mobile applications.
It is therefore necessary to study its interfaces, development
methods and API of the machine.

\subsection{The Application Platform}
Karotz application (called ``Appz'') are installed through an
online platform located on the Karotz web site. They can be
launched on the Karotz itself either through a scheduler, voice
commands or an RFID chip. 

As for launching the BLOPP app, the times for the scheduler must
be manually set through the Karotz web site, so it cannot be 
used for notifications directly. The best option for the BLOPP 
implementation would thererfore be to set a scheduler to start
every day at 00:00 and stop every day at 23:59. This way it can
be ensured that the application is always running, updating
itself with medications, status and times, and a timer can be
used to schedule notifications.

\subsection{The two APIs}
The Karotz can be programmed in two different ways: either
through a web REST framework, with JavaScript that runs as an
embedded program on the robot itself.

The requirement that a REST program would have to be hosted
somewhere, combined with the fact that an embedded program 
provides more flexibility in terms of local storage to limit the
amount of information sent over a network makes the JavaScript
framework a more suitable choice for the BLOPP project.

\subsection{Karotz Output Channels}
The Karotz has a few ways of providing output to an end-user. It
can be asked to
\begin{itemize}
    \item play sound files;
    \item move its ears;
    \item speak through a TTS (text-to-speach) engine;
    \item illuminate its stomach with different colors;
    and
    \item communicate through the internet with HTTP GET and POST 
          methods.
\end{itemize}

For providing user commands, TTS could be an option if the engine
supported Norwegian, but since the language options are limited to
English, French, German and Spanish, speech will have to be
created by recording sound files and playing them with the
multimedia engine.