\appendix{Risk Measurement}

This appendix consists of an analysis of what actions will be taken in case of any unexpected factors taking place.\\

\paragraph{IR1:Illness among the developer team}
The developer team has secured themselves against the event of illness among the 
developer team in the following ways. Overlapping roles; all development roles have a 
backup person, and the team members have good communication among themselves in 
case of illness. This way no one is too essential to the project. All tasks are also 
fairly independent and that way easy to be overtaken by other members of the team.

\paragraph{IR2:Long-term leave. A group members takes a long-term leave}
The developer team has secured themselves against the event of members taking a 
long-term leave in the following ways. As mentioned earlier, the team has overlapping 
roles. It is also made clear by the team members that it is expected to work in the 
hours they're gone due to long-term leave. There's a common understanding that ''There's no I in the team".

\paragraph{Internal conflicts among the developer team}
The developer team has secured themselves against the event of internal conflicts 
in the following ways. The team consists of five members, which makes it easier 
to have decisions based on the majority's opinion. Should conflicts occur, the team 
may also confer with the advisor or even the customer, if suitable. Should there still 
be a problem, team members are told to swallow their pride and hash it out in a 
rock-scissor-paper contest. In a worst case scenario, the team will confer with the 
advisor about a member taking leave of the project, but this is the last option in any solution.

\paragraph{Deadlines not being reached}
The developer team has secured themselves against the event of deadlines not being 
reached in the following ways. Lowering the expectations before starting on something. 
No one is a superman, and should only take on work they may finish in time. That being 
said, it is demanded that team members do their share. The team members are prepared 
for working during the weekend in order to finish their tasks. 

\paragraph{Overworked group members}
This is the most dangerous and a likely risk to occur. All team members have many other 
courses to attend, part-time jobs and other tasks they are expected to finish, leading 
to potentially a lot to do. All team members are expected to prioritize the project over 
other activities. Should team members be overworked, they may take lighter week of  workload, 
in order to come back on track. It is although expected that the team members work extra 
the following weeks in order to come back on track. 

\paragraph{Group members leaves project}
The course is mandatory for all computer science masters degree students. In any case team 
members may leave and take the course later. In case of team members leaving, the team 
has secured the project by overlapping roles, and expectation-management. Should a member 
leave, the customer will have to understand that with less manpower, less work may be done. 


\paragraph{Group members showing up late}
The team has secured themselves against the risk of team members showing up late in the 
following ways. No meetings are held to early in the morning. Meetings are held at 0900 
as the earliest, which is an acceptable time to start the workday. It is a common understanding 
that showing up late is not okay, and should it occur to often, the team member will have 
to bake a cake for the next team meeting. 


\paragraph{Lack of knowledge or abilities}
The team expects to meet situations where they may not have the needed knowledge or abilities. 
Therefore the team has team meeting every other day, where day may ask for help. The team also 
has a mailing list and a private facebook group where they may ask for help. The team also 
keeps a list of useful internet sites they may visit to ask for help (in.e. www.stackoverflow.com). \\ \\


The team has a some external risk factors. These are analysed in the following section. \\

\paragraph{ER1:External Conflicts. One or more of the group members is in conflict with the customer or the group advisor.}
The team meets with the customer and the advisor once a week. There is always a potential 
of conflicts in settings where decision making is done. If a conflict should arise, the 
actual team member may not need to attend the next customer or advisor meeting, since not 
all team members are needed at all meetings. If problems should occur, the team members will 
confer with the advisor or customer on what to do about the situation. In a worst case scenario, 
the team member in conflict will be asked to take leave. This being a last way out, and not in any way preferred. 

\paragraph{ER2:Customer changes requirements}
The customer may be indecesive or gives contradicting messages concerning the requirements or 
conditions. The customer may want changes to the project. To take care of this, the team has 
meetings with the customer every week. The customer may not give a contradicting message during 
a sprint, but may give input during a sprint demo meeting. The customer is expected to change the 
requirements during the run of the project, but they are aware of the results this may have on the 
final result. To many changes will lead to much time being used on making changes, rather than 
making improvements. 

\paragraph{ER3: Lack of input from customer. There is insufficient input form the customer regarding the development process}
This risk is not very likely, since the customer is very active and wants to be as much 
involved as possible. Should this happen anyway, the team will make sure to force the customer 
to give feedback during costumer meetings. If needed, the advisor will be involved, in order to 
make the customer understand the importance of feedback.

\paragraph{Tools fail. Software tools stop working or are outdated}
The development team has secured themselves against this risk in the following ways. Making 
sure a new release is stable. Should there be an update for a software tool available, the 
team should make sure that the release is stable and well working. This being said, the team 
should always use the latest stable release. \\
The team also try to make the documentation and source code as independent as possible of tools. 
There shpuld always be a backup tool available, in case of tools not working.

\paragraph{Loss of data. Data connected to the project is lost, or is unavailable for a period of time}
All data is backed up at Github. This way there is an online storage in which the code is stored. 
In case of Github servers being unavailable, the team also has their own version of the source code 
at their local computer, making it available to work independently of Github. Github has very little 
downtime, and this is not expected to happen. 


\paragraph{One of the customers takes a long-term leave}
One or more of the customers may take a long-term leave. Since the customer group consists of four 
people, two of them in full-time jobs, this is not very likely.
