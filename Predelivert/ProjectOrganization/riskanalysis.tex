\chapter{Risk Analysis}

This section contains a risk analysis done before the project was started. The 
analysis helped the team detect the most relevant risks for the project, in order 
to be prepared if such a problem should occur. Allowing the team to make som pre-emptive 
measures as well as draw up strategies for handling a number of possible situations. 
In addition to a listing of internal and external risks, this section contains a SWOT 
analysis for a deeper understanding of the different relations in the project, both internal and external.

\subsubsection{Internal Risks}
The list below contains issues that the group identified as possible internal risks for the project.

\begin{landscape}
\begin{table}[h]
\begin{tabular}{  c 	 	l 	 	p{4.0cm} 	 	p{4.0cm} 	  	c 	 	p{4.0cm} 	 	l }
\hline
  \# & Activity 	& Risk factor 	& Conseqences 	& Prob.& Strategy and action 							& Responsible \\
\hline
  IR1 & All		& Illness among the developer team.  	& M -- Deacrease of productivity. Lack of expertise.	& M 	& Share information online. The sick developer keeps updated. 	& All \\
\hline
  IR2 & All		& Long-term leave. A group member takes a long-term leave.  & M -- Reduced capacity.  & L	&  Plan what the member shall do before he is leavng.		& The Member \\
\hline
  IR3 & All		& Internal conflicts among the developer team.		& M -- Bad mood. Less work effort.  & H	&  Bring it up, and handle it right away. Use the advisor. 		& All \\
\hline
  IR4 & All		& Deadlines not being reached.			 	& H -- Unfinished work.  & M	&  Predict work load and set project boundaries. 	& Project leader \\
\hline
  IR5 & All		& Group members busy with other courses.		 & L -- Reduced capacity & H	&  Common Google Calendar to plan meetings. The member finds another time to do the work. 	& All \\
\hline
  IR6 & All		& Group members leaves project.		 	& H -- Reduced capacity & L	&  Reduce the project boundaries.	& All \\
\hline
  IR7 & Meetings		& Group members showing up late.	 	& L -- Reduced capacity & H	&  The group member works more next time. Penalties. 	& All \\
\hline
  IR8 & Implementation		& Lack of knowledge or abilities.	 	& L -- More time goes to getting knowledge. & M & Getting the knowledge. 	& All \\
\end{tabular}
\caption{Internal Risks}
\label{tab:riskanalysis}
\end{table}
\end{landscape}

\subsubsection{External Risks}
The list below contains issues that the group identified as possible external risks to the project.

\begin{landscape}
\begin{table}[h]
\begin{tabular}{  c 	 	p{1.8cm} 	 	p{4.0cm} 	 	p{4.0cm} 	  	c 	 	p{4.0cm} 	 	p{1.8cm} }
\hline
  \# & Activity 	& Risk factor 	& Conseqences 	& Prob.& Strategy and action 							& Responsible \\
\hline
  ER1 & All		& External conflicts. One or more of the group members is in conflict with the customer or the group advisor & M -- May lead to bad communication, lack of feedback etc. & L 	& Bring it up and handle it right away. If customer contact is in a really bad conflict, switch customer contact.	& Customer contact \\
\hline
  ER2 & Design and implementation	& Customer changes requirements. The customer is very decisive and demanding. & H -- May stress the developers and and make them confused on what to prioritize.  & H	&  Force the customer to prioritize tasks. Compromise possible solutions. 	& Development team and customer \\
\hline
  ER3 & All		& There is insufficient input from the customer regarding the development process.  & M -- May lead to expectencies not being met.  & L &  Force input from customer.	& Development team and customer \\
\hline
  ER4 & Development	& Tools fail. Software tools stop working or are outdated. 	& M -- Development process halted.  & L	&  Do other work. For instance report writing, testing and refactoring. Find other solutions to the problem. & All \\
\hline
  ER5 & Development & Loss of data. Data connected to the project is lost, or is unavailable for a period of time.   & H -- Development put back in time. & L &  Prioritize tasks according to remaining time. 	& All \\
\hline
  ER6 & Meetings, Feedback	& One of the customers takes a long-term leave. & L -- May delay feedback, input and access to resources. & L & Require more from the other customer contacts. 	& The customer \\
\hline
  ER7 & Development & Karotz API not working at all. & M -- May lead to removal of this feature for the prototype. & L	&  Focus on other parts of the applications.  & Development team. \\
\hline
  ER8 & Development		& Database not working at all.  & H -- May lead to hardcoding of all features for the prototype. & L & Hardcode necessary parts.	& Development team \\
\hline
  ER9 & Development		& Android devices stop working  & L -- May delay development since emulator has low performance. & L & The group member must switch to emulator. 	& Group member \\
\end{tabular}
\caption{External Risks}
\label{tab:externalrisks}
\end{table}
\end{landscape}

\subsubsection{SWOT analysis}

The following section contains an analysis upon Strengths, Weaknesses, Oppurtunities and Threats 
to the project. The analysis is used as a strategic planning method which analyses the internal 
and external factors in a group. The internal factors are strengths and weaknesses while 
oppurtunities and threats are external factors. 
The SWOT analysis' intended use is to get an overview of the internal and external factors in the 
beginning of the project. Afterwards, the analysis was used to map which parts of the projects 
should be relied upon the most, and which opportunities and strengths can help the progress of 
the project the most. 
Keeping the high-risk parts of the project under close watch will help the team catch problems 
before they evolve. In cases where a diversion was required, the SWOT analysis should support the 
team. Appendix Y has a detailed overview of the internal and external risks that have been analysed. 
%TODO update references

\subsubsection{Strengths}

Communication and knowledge are the two greatest strengths in the project. The fact that all 
team members have Norwegian as their native language, and are fairly competent in English, makes 
all communication and reporting easier. Deciding to the reporting and programming in English, 
while all other means of communication in Norwegian lead to fewer misunderstandings and made 
it easier to help each other out when problems occured. Discussions were also more valuable 
since everyone were able to participate without the fear of missing out due to lack of understand 
foreign languages. 
When it comes to level of knowledge, all group members are fourth year Computer Science students. 
This implies that even though the group members are taking different paths as to which 
specialization they are following for their masters degree, they all have a common background. 
It is therefore to be expected that everyone can participate in both coding and writing. The 
consequences of somebody falling ill are low, since they may be replaced with another member of 
the team. This is based on the general cooperation of the team.
The technology is being shared between all team members and every team member will take part in each part of the development. 
In addition to the basic knowledge, team members have chosen their own combination of subjects. 
This makes the team more capable of solving a broad array of tasks and come up with good solutions. 
Another strength in the group is that everyone has experience from previous projects, some more 
than others, but everyone has been involved in IT related projects. This provides the team with 
the experience needed to avoid some common pitfalls and get a decent start on the projects. 
The applications is to be used by patients with chronic illnesses, such as asthma. Since there are 
members of the group with asthma, the group might find it easier to take the end users role into account.


\subsubsection{Weaknesses}

The team chose to use the document markup language LATEX, even though none of the members 
had used it before. This lead to some problems during the start of the reporting. The team 
was able to find many different guidelines, research and tutorials to use LATEX, which made 
it a lot easier to use, but it took time to learn and configure everything. 
For most people, money is a huge motivating factor. Since this is a university course, 
the entire group will be working for free, making a product someone else may profit on. 
This meant that the developers had to find some other form of motivation. 
The group is also given a single grade, based on the overall achievements and results made 
by the team as one unit. This resulted in the biggest possible gains being experience, 
knowledge, relations to customers and the final grade. As students, the grade is usually the 
most motivating factor we get from a course. With a common grade for the group, there is a 
risk that expectations for the final grade might differ among the team members, and that 
some members will settle for a lower grade than others. It could prove difficult to get 
everyone working equally hard and as a result, some group members might become frustrated. 
The team attempted to fight this risk factor by introducing this weakness and discussing what 
each member wished to gain from the project.

\subsubsection{Opportunities}

Write something useful about NAAF and their effect on our work.
Since Ole Andreas Alsos is working in BEKK Consulting on a daily basis, the team has been 
given opportunities to talk to professional developers for short questions about development, 
methods, tests and other useful themes. 
The concept behind BLOPP has a lot of potential users. There is a significant need for a 
technological breakthrough in this area, but in order to avoid spending money on a poor solution, 
BLOPP would to do this as a low-cost project first. Should this project result in a success, 
NAAF can apply for a financial support from the Department of Health to develop the project 
further. The applications may also be useful for other people with chronical diseases.


\subsubsection{Threats}

When making a new product, it is not always clear what the product is going to solve and how it 
is going to do it. Neither are the opportunities and the limitations. Therefore it is very likely 
that the product and requirements are going to change during the development process. It is critical 
that the team makes room for unexpected changes and that adapting to changes is made easy. 
To manage this risk, a good working relationship with BLOPP is necessary. Product tests and 
demonstrations need to be done iteratively with both users and customers. Failing to do this will 
be a huge threat to the project and therefore, communication and collaboration will be important. 