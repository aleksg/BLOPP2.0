\section{Version Control}

Version control is an essential part of software development in a group. Code
sharing and collaboration has to be done effectively to succeed when there are
several people working on the same documents. As a result, the version control
system is essential. We quickly decided on GIT3 for our project. The decision
was mainly made on past experience with GIT3 and other version control tools.
By choosing a system that someone had prior experience with, a lot of time was
saved when it came to setting up the working environment and learning how to
use it. Each file has its own version number which increments each time someone
commits changes to the file. All developers will have access to these files, so
it will be easy to share. GIT3 allows you to roll back to an earlier commit, in
case of broken code or

The source will have two main branches for the source code. The
``Master''-branch
will only contain source code which is fully implemented and working. It will
mainly be used as a backup and/or reset point and for customer demonstrations. 
The “DEV” branch will contain source code which is under development. This may
contain buggy code or code not working.

The development team may branch further. If done, this is done to avoid
conflicts in specific parts of the source code, and will the branches will be short-lived.

We will also use Git to keep track of this report. For the report, all commits shall be 
on the ``MASTER"-branch.