\section{Coding Style}
The applications uses Android as platform. As a consequence, most code will be
written in Java. The coding style mentioned here only applies to code written
in  Java. 

\subsection{Package conventions}
The package names for GAPP will be on the format ``no.blopp.app.X'', where X (lower case) describes the content of the package. 
The package names for CAPP will be on the format ``no.blopp.app.med.Y'', where Y(lower case) describes the content of the package.
For instance no.blopp.app.activities, or no.blopp.app.med.activities. 

\subsection{Indentation}
All statements, conditional expressions, function declarations and class
declarations shall be written on a separate line. 

\subsection{Curly Brackets}
The opening curly bracket following a function or class declaration shall appear
on the line below the declaration, with equal indentation as the function
declaration. The closing bracket shall also appear with this indentation, on
the line below the code block.

\subsection{Naming Conventions}
The following naming conventions will be used when writing code in Java. We will
use an ``I'' in front of an interfacename, to better recognize these. That is
the only place we have that kind of convention. Table \ref{tab:javaNamingConventions}
shows an overview of the naming conventions for Java code.

\begin{table}
	\begin{center}
		\begin{tabular}{|p{4cm}|p{4cm}|}   
			\hline      
			\bf{Type} & \bf{Convention} \\ 
			\hline
				Local variable & lowerCamelCase \\     
			\hline
			 	Class & UpperCamelCase \\
			\hline
			 	Interface & \emph{I}UpperCamelCase \\
			\hline
			 	Constants & UPPERCASE \\
			\hline
		\end{tabular}
	\end{center}
	\caption{Naming convention}
	\label{tab:javaNamingConventions}
\end{table}

\subsection{Android views}
The android framework has a lot of predefined elements like buttons, textviews
and layouts. When creating these elements, they have an id referenced by
``android:id= my\_id''. These id's will be a combination of type a\_b. ``a'' will be
a constructive word describing the element. ``b'' will be the component. ``a''
will be written in lowercase only, seperated with underscore. ``b'' will be in
lowerCamelCase. Examples of this can be ``back\_to\_menu\_button'',
``date\_textView'' and so on. The reason for this is that android automatically
generates R.java, containing these id's. It should be easy to know excactly
which id you are looking for when calling the method findViewById(id).
