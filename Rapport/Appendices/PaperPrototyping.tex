\chapter{Paper Prototype}
\section{About paper prototyping}
This section describes the process of paper prototyping. 
A paper prototype is a low-fidelity prototype made out of paper, post-it notes or similar material. 
The idea is making a layout in paper, to test the flow of the program, the layout and design to 
root out early design flaws. By making this out of paper, a low-cost prototype is ensured. The 
prototype is then tested by an external test person, using different predetermined scenarios. 

Usability testing using paper prototypes does have it's limitations. The paper prototype makes 
it difficult to simulate animations, sounds, scrolling and silimar functionalities. The test 
person must also be able to imagine that the paper prototype is a real program, even though it's 
made out of paper.  

The usability testing done using the paper prototype is explained in Section \ref{sec:paperprototypetest}.
%Needs references, fixed

\section{Usability Testing with a paper prototype}

\subsection{Testprocedures}
The usability test is done by making the testperson completing a series of tasks with the help 
of the paper prototype. The tasks must be very specific, meaning they must be specified in a way 
which makes the testperson search for specific information, press specific elements on screen 
or similar. The task shall be realistic and representative for the normal use of the system.

An example of such a task may be: ``You wish to change the health state of Ole Olesen from Good 
to Bad. Please do this via the application''.

\subsection{The testpersons tasks}
The testperson shall solve the tasks given, while he/she speaks out loud what he/she is doing and 
why. The reason for this is allowing the designers to understand the thoughts and the mindset of 
the testperson's user experience with the prototype.

\subsection{The testgroups tasks}
The team leading the test shall have clearly defined tasks under the test. Testleader gives 
instructions to the person doing the test and tells what is happening.

``The Wizard of Oz'' is responsible for changing out the ``frames'' (papers representing frames) 
when the user interacts with the paper prototype.

Observators don't take part in the testing, but observe and take notes throughout the testing.

Neither the testleader, the Wizard of Oz or the observators may answer questions during the testing.
