\chapter{Evaluation}
\label{chap:evaluation}

This chapter contains an evaluation of several aspects of the project. The evaluation discusses the work process the group went through in this project. This includes all work connected to development of the system, how the development methodology and routines worked and how the group experienced the work load of this project. Next follows a description of the state of the project at delivery and a discussion of why it was in this state. The chapter ends with a conclusion of how the group has experienced the project, and how satisfied the group has been with the learning process for this task, and with the way the project was organized by the coordinators.
Section \ref{sec:workprocess} evaluated the work process of the project. This includes the development process, work routines and an evaluation of the work load. In Section \ref{sec:finalproduct} the final product is reviewed. This includes a discussion of whether or not the product is finished, and why it is delivered as it is. At last, Section \ref{sec:concludingremarks} gives a conclusion on the evaluation.


\section{Work Process}
\label{sec:workprocess}
During  the project there have been several aspects in the work process that is worth a mention. Even though all of us have prior experience in team work and development, we had to learn to work together as a team. As a result, the project could have been done differently. 


\subsection{Development Methodology}
During the beginning of the project, we were very eager to use Scrum as a method of development. The reason behind this may be the fact that most IT-companies tend to speak very highly of Scrum and how well it works. Many of the team members had never tried out Scrum before, and were therefore interested in learning how this was done.

During the project we understood early on that Scrum was not as optimal as we first assumed. There were many factors resulting in Scrum not being optimal. One of the major artifacts of Scrum is the daily stand up. This was very difficult to carry out, since each team member had their own lecture plan, and it was therefore difficult to find suitable times for a scrum meeting, even though it only lasts for 10 minutes. We tried to follow the plan of doing semi-daily stand ups as strictly as possible, but many times we used AgileZen or IM clients to update each other. 

The use of a scrumboard is yet another important artifact of Scrum. Since it was not possible to find a permanent workroom, we used an online scrumboard via AgileZen, in addition to a spreadsheet in Google Drive. This worked fairly well, but it would have been better to have a physical scrumboard, since it would remind and motivate the team in a higher degree. 

The customer was very involved in the process, regarding functionality, prioritizing tasks and giving feedback on results. This helped us focus on what was important to finish. We decided early on a 14 days length. The main reason for this was the fear of wasting to much time on planning, reporting and retrospective if the sprints were only a week long. Sprints of this length lead to us having to take feedback and design for change in the middle of a sprint, since we met with the customer each week. This was done in an orderly fashion, by us not planning to much in detail for what functionality was to be implemented in each section, but rather what section was up for improvement during each sprint. We were in a way more agile than a usual Scrum project would be. Being more agile turned out for the better, since we quickly eliminated unnecessary tasks, though it was stressful at times, since the customers came with new demands during the sprints. 

One aspect we failed at was having a working application at the end of each sprint. At the end of each sprint there was always some parts of the applications that did not work properly, and instead of removing them for the demo, we didn't show these parts at the demo. This worked out OK, but is not the proper way to do things, since the customer may be confused during demos, and ask about nonfunctional parts of the system. Usually in Scrum either functionality is complete, or it's not in the demo application at all. Meaning it's not possible to find it when searching through a demo application.

\subsection{Development Process}
The development process was a dominating part of this project. We used about 50\% of our time on programming, which is normal. The reason for this amount spent on programming have many factors. The main reason was that we wanted to spend as much time as possible developing a working prototype, and not delivering a half-finished product. At no point did we rationate the hours available, we rather kept track of what tasks were to be finished, and spent our time thereafter. 

The customer was very present during the entire process, and gave much feedback on the functionality implemented, changes in requirements and priorities. The requirements were changed after every customer meeting. This resulted in some functionality being removed, even though it was already implemented. Throwing out some functionality is normal, since opinions may change when seeing the final product, or problems may occur during development. 



\subsection{Work Load}
The work load of this project has been intense, but it has resulted in a huge amount of learning and experiences in software development.
The focus of this project had two main goals, do the work and document the work done. Since the stakeholder for each of the goals are two separate groups, the workload is not necessarily perfectly balanced. The customer wanted us to implement more functionality, while the our advisor told us to write a better report. As stated in the course description, each team member should use 25 hours a week, to a total of 325 hours on the project. This was a constant struggle, as all team members had at least two other courses to attend, along with exercises to finish in these courses. This was a constant stress factor throughout the project.

The exercises of other courses where at times very time-consuming, effecting our effort on the project. When another course had an exercise up for delivery, we had problems filling the hours demanded for that certain week. This resulted in a very uneven work effort from each of us. A more even work flow would have been more preferable and much crunch-time would have been avoided. 

Since all team members had different lecture plans it was hard to coordinate when to work together. Working together is an absolute necessity when programming, and this could have been planned better by the lecturers.

The development lasted until there were ten days until delivery of the report. For the last ten days the focus was directed towards writing the report and documenting the code. 

The final source code consisted of almost 11 000 lines of code, which is a lot considering the time and resources we had available. 


\section{The Final Product}
\label{sec:finalproduct}
The goal of this project was to deliver a fully functional prototype, which later could be used in order to launch a full-scale product. Unfortunately the final product did not include all the functionality wanted from the customer. Yet, we are of the opinion that this is a very functional and well working prototype, and we are very curious as to what this prototype will lead to in the future. 

\subsubsection{Karotz implementation}
The implementation of the Karotz in the project was a good idea, on paper, and that's about it. The children we tested on found the Karotz funny and nice, and by moving the focus away from the medicine and towards the Karotz, it made the mask the children use for taking their medicine seem less frightening. 

To work with the Karotz was easier said than done. The documentation of the Karotz API is written in French. 
A huge problem with the Karotz was connecting it to the internet. In order to make it run the program we wrote for it, we had two possibilities, either deploy to the Karotz website or run the program locally. Deploying the application to the Karotz website was not an option since it would have to wait for approval, and there was no time estimate for how long it would take. When running it locally the Karotz has to be connected to the same network as the computer running the program. Also there is not many possibilities for storing any information on the Karotz and the documentation given by the Karotz documentation was faulty and did not explain how to store programs on the Karotz. This resulted in a solution where the program had to be downloaded for each repeated run. Another huge problem is that the Karotz will update itself every \~ 30 minutes, meaning that the program running on the Karotz will be deleted and will not start itself again, making the use of the Karotz a high-maintenance task. 

We also had some problem with the developer website with the documentation of the API. The website was unavailable for a week, during our project. When we reached out to their support desk, they had no time estimates for when it would be back up, leaving us out in the cold. 

All in all the implementation of a robot toy is a good idea, since it may appeal to children and make it more enjoyable for them to take their medicine. Using Karotz for this task is not a good idea, and should be avoided. Unfortunately there are very few alternatives as to this. There is reason to believe that the cost-benefit ratio for an end user will be so low that they would have little benefit from using a Karotz. 

\section{Functional Requirements completed}
\label{sec:frcompleted}
Table \ref{tab:functionalrequirementscompleted} shows an overview of the functional requirements stated in Section \ref{sec:functionalRequirements}, and whether they are completed or not.
As one may notice, all functionality with high priority is completed. The parts that are not completed can be classified as ``nice to have''-features, but are not vital 
for the prototype.
\begin{table}
\centering
\begin{sideways}
\begin{tabular}{|p{5.0cm} | l | l | p{9.5cm} |}
\hline
Functional Requirement & Priority & Completed & Comments \\
\hline
PFR 1 - Medication plan & High & Completed & Supports simple medication plans. Does not support several medicines that is to be taken at the same time. One minute delay is a potential workaround. \\
\hline
PFR 2 - Notifications & High & Completed & An alarm is goes off once it is time to take medicine. The email-notification was not implemented. \\
\hline
PFR 2.1 - Settings for notifications & Medium & Not completed & Due to short time, and because we needed access to file system to find ringtones. \\
\hline
PFR 2.2 - Notification to change conditions & Medium &  Not completed & Did not have time, although there is not a whole lot of work to extend the application with this functionality. \\
\hline 
PFR 3 - Families & Low & Not completed & - \\
\hline
PFR 4 - Guidelines & High & Completed & - \\
\hline
PFR 4.1 - Guidelines from NAAF & Medium & Not completed & - \\
\hline  
PFR 5 - Keep records of condition & High  & Completed & - \\
\hline
PFR 6 - Pollen forecast & Medium & Somewhat completed & NAAF's pollen cast is not running at the moment. We replicated the XML-structure for the purpose of the prototype.\\
\hline  
PFR 7 - Screen sizes & Low  & Not completed & Only supports screen sizes at 480x800 at the time being. Easily extended, but needs scaling of pictures to fit the screen. \\
\hline
\hline
CFR 1 - Distraction & High & Completed & - \\
\hline
CFR 2 - Rewards & High & Completed & - \\
\hline
CFR 2.1 - Rewards & Low & Somewhat completed & - \\
\hline
CFR 3 - Screen sizes & Low & Not completed & Refer PFR 7. \\
\hline 
CFR 4 - Avatar & Low & Not completed & The customer had a hard time settling on our gamification concept. In the end there was not enough time. In cooperation with the customer, we decided to scrap the idea. \\
\hline 
CFR 5 - Child friendly instructions & High & Completed & Image gallery that shows very simple drawings of how to take a medicine. \\
\hline
\hline
KFR 1 - Notification & High & Completed & - \\
\hline
KFR 2 - Distraction & High & Completed & Children needs to interact with the karotz when taking a medicine. This helps the user get distracted. \\
\hline 
KFR 3 - Reward & High &  Completed & - \\
\hline
KFR 4 - Register use of medicine & Medium & Completed & - \\
\hline 
KFR 5 - Logging & Medium & Completed & - \\
\hline
\end{tabular}
\end{sideways}
\label{tab:functionalrequirementscompleted}
\caption[Functional requirements completion]{The original functional requirements, whether they are successfully implemented or not, and comments}
\end{table}
\clearpage{} %To avoid breaking up concluding remarks

\section{Concluding Remarks}
\label{sec:concludingremarks}
The project has been engaging, educational, stressful and challenging. In retrospect, we all agreed that the experiences has been worth the effort and time we have spent. The weight of what we learned in terms of project management, programming, team-work and customer relations and system development has been huge. Even though the project lead to team members having less time for other courses, we are in the opinion that the time spent has been worth it. 
Regarding the final product we are proud of what we have delivered, and are very curious about the future of the BLOPP project. The domain of the project has made us feel that we have had the possibility to make changes and actually help children with asthma and their parents.
If we would have done the project one more time, we would have done some things differently. 
First, we would have designed CAPP for multiple users from the beginning. We started implementing CAPP without having a plan for how to implement support for multiple children. At a later stage, when support for multiple users was up for discussion, we had to decline the idea, since it would take too much time to refactor all code already written. 
We should have arranged programming sessions from the beginning. When working as a team, it is essential to be in close proximity, in order to make communication more effective.
The shop functionality was left in the cold for too long. The customer was not certain if they wanted the shop, and at one point the abandoned the shop. We are in the opinion that the shop would have been a very cool idea, if done right. If we had decided to make a shop from the beginning we believe it would result in a great element for motivating children.