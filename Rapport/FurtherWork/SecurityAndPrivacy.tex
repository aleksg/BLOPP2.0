
\subsection{Security and privacy}
\label{sec:sec-and-privacy}
The application is, as mentioned, using a webservice to access database records. This webservice is currently not using any form of encryption or protection.
Ideally the webservice should be using the HTTPS protocol instead of HTTP, and have proper password protection, to avoid that records of childrens medication history is publicly
available. This will have some issues on performance, but if a proper caching system is implemented as mentioned in Section \ref{sec:wifi-caching}, it would be hardly noticable.


% We are at the time being not using any sort of personal information like SSN, street adress and so on. Though, we have it as a column in our database in case this is useful for further work.
The only ``identifier'' we are currently using is the database ``id'' of a child, which is an automatically incremented number, and can be seen as a random identifier for a child, meaning that it does
not have anything to do with a child at a personal level (not a username, name, emailaddress, ssn, etc.). This raises a question whether or not privacy is a real issue here. 
For one to actually know which child has been taking which medicines, one must either know the stored database id for a child, or hack the system and get access to the name. 
From our perspective, privacy is not a real issue that needs to be taken care of at the time being. 


If however, the application in the future becomes more personalized as mentioned in Section (Minor improvements), these are issues that needs to be taken seriously, as medical history is considered sensitive information. 