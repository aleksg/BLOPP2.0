\chapter{Further Work}
\label{chap:Further Work}

This chapter gives an overview of some of the ideas both the customer and the developers had for further development of the application. This includes a description of further development, analysis of the user groups and work towards NAAF and the health department.
The main part of the work to be done after the end of this project is connected to requirements that has been taken out of this project due to limitation of time and resources. Other issues remaining is connected to the security and privacy of the patient's treatment log and storing sensitive information.
Section X.XX lists the overall requirements that have not been implemented during the project. These requirements has either been requested early in the process of have been brought up during discussions and meetings with the stakeholders. 

% Here goes the major potential improvements such as privacy/security, rewardsystem and connection
\section{Improvements}



\subsection{Wifi access and caching of database records}
\label{sec:wifi-caching}
The application is currently very dependent on having network access in order to run at all. We are making a lot of HTTP requests, and we never cache the results. This introduces
performance issues. This could be avoided by introducing a caching service. The application should store 
information about when the last database update occured, and update the cached information appropriately. This is a part of the system that would take a lot of time to implement, 
and we just did not have enough time.  


Once the device loses its internet connection, the application should use these cached results to update the different screens, for example the log. 
This caching service should also cache information about treatments that is done when the device is not connected. For instance, if a child takes a 
medication, the device stores this information. Once the device connects to internet, the applications should push these changes to the webservice, 
which, in turn, updates the database records, and thus we have applications that is not so ``tied up'' to a device's internet access. This could also 
introduce the possibility to never connect to internet at all, by just using the stored data.   


  

\subsection{Security and privacy}
\label{sec:sec-and-privacy}
The application is, as mentioned, using a webservice to access database records. This webservice is currently not using any form of encryption or protection.
Ideally the webservice should be using the HTTPS protocol instead of HTTP, and have proper password protection, to avoid that records of childrens medication history is publicly
available. This will have some issues on performance, but if a proper caching system is implemented as mentioned in Section \ref{sec:wifi-caching}, it would be hardly noticable.


% We are at the time being not using any sort of personal information like SSN, street adress and so on. Though, we have it as a column in our database in case this is useful for further work.
The only ``identifier'' we are currently using is the database ``id'' of a child, which is an automatically incremented number, and can be seen as a random identifier for a child, meaning that it does
not have anything to do with a child at a personal level (not a username, name, emailaddress, ssn, etc.). This raises a question whether or not privacy is a real issue here. 
For one to actually know which child has been taking which medicines, one must either know the stored database id for a child, or hack the system and get access to the name. 
From our perspective, privacy is not a real issue that needs to be taken care of at the time being. 


If however, the application in the future becomes more personalized as mentioned in Section (Minor improvements), these are issues that needs to be taken seriously, as medical history is considered sensitive information. 


\subsection{Rewardsystem}
The childrens application (CAPP) are all about changing the childrens view of medication to something positive. It shall be a motivation for the children to take their medication. It is therefore an important task to entertain them and give them some form of reward when they take their medication. As for now, we have given stars to the child after taken medication. This stars are in a treasure chest where the child can see how many stars it has. This is an easy reward and worked fairly well under the user tests. However, it may be boring over time. 

The initial idea was to have a shop where the children could buy clothes and stuff to their avatar. This was not implemented due to time concerns. There are also possible to take this to the real world. E.g. that the child gets a lollipop for evrey 10th star. 
 
There are a endless line of opportunities for this reward system, and we chose the easiest part to have something to test. Hanne, who writes a thesis about this project has developed a potential evlolvement of this reward system. This includes an interactive storyline where the story evolves based on the childs choices. More on this in her report \ref{XXX}. 

\subsection{User testing of the guardian application}
The GAPP has not yet been user tested on actual parents of asthmatic children. This has to be done to get an understanding of how they interact with the system, and to get knowledge about what they think of an application of this type. This is a system to make it easier for the guardians to give their children medications. It doesn't matter what the children thinks if the guardians only think this would slow down the process, because the system will never be used if the guardians don't like it.

% Here goes the minor ideas and improvements for further work

\section{Further ideas and minor improvements}


\begin{description}

\item[Webinterface] The doctors may prefer to set up the users medication plans through a webinterface on their computers. 

\item[Other devices] The application are fitted for a phone running the Android operating system. For the future it should also be scalable to tablets. There may be more interesting for a child to work on a tablet than a phone. There will also be much more space for content. This extra space gives greater potential of the reward system. It should also be available on other operating systems than Android, e.g. iOS or Windows Phone. This will improve the availiability for the users, not limiting them to Android phones. 

\item[Overall graphical design] The priorities have been to make the major functionality work. We have used lots of time making the applications understandable and easy to use, but there is still a great potential in making the applications interaction design better. 

\item[Personalize the system] The application may be more personalized. E.g. "It's time to take medication" could be "It's time to take medication, Eric". By involving the users name more in the system, they may feel more appreciated. 

\end{description}