
\subsection{Wifi access and caching of database records}
\label{sec:wifi-caching}
The application is currently very dependent on having network access in order to run at all. We are making a lot of HTTP requests, and we never cache the results. This introduces
performance issues. This could be avoided by introducing a caching service. The application should store 
information about when the last database update occured, and update the cached information appropriately. This is a part of the system that would take a lot of time to implement, 
and we just did not have enough time.  


Once the device loses its internet connection, the application should use these cached results to update the different screens, for example the log. 
This caching service should also cache information about treatments that is done when the device is not connected. For instance, if a child takes a 
medication, the device stores this information. Once the device connects to internet, the applications should push these changes to the webservice, 
which, in turn, updates the database records, and thus we have applications that is not so ``tied up'' to a device's internet access. This could also 
introduce the possibility to never connect to internet at all, by just using the stored data.   


  