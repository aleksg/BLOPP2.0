\section{Project Information}
\label{sec:prosec}
\subsection{Project Name}
The name of this project is ``BLOPP", and was decided by the customer. BLOPP stands for ``Barns 
LegemiddelOPPlevelse'' (``Children's experience with medication'').  

\subsection{Background}

Many children today have to take inhalation medicines because of chronic or
acute lung disease such as asthma. Children often find it difficult to use
the medication correctly, boring or even scary to take them, which means they
might object or forget to take them. Parents also sometimes
apply the medication incorrectly, apply the wrong treatment, or even forget to
give the medication to their
children. This may lead to reduced effect of the medication, and the lung disease
may worsen and last longer, causing increased pressure on the public health
services, increased health related cost and lost working hours for the parents.




\subsection{The task}
Our task was to implement two Android applications, one application for the parents, Guardian Application (GAPP), 
and one application for children, Children Application (CAPP). In addition, an application for the Karotz platform 
should be created to assist and to an extent substitute the GAPP and CAPP mobile applications. 

The high level functional requirements for these applications are to be found in Table 
\ref{tab:highfuncitionalrequirements}. The functional requirements for all applications are described in more detail 
in Section \ref{sec:functionalRequirements}

\begin{table}
\centering
	\begin{tabular}{p{1.5cm}|p{12cm}}
		\hline
		\bf{\#} & \bf{Description} \\  \hline \\

		GHR1 & The application must alert the parent(s) when it is time for a medication/treatment for their child. \\ \\
		GHR2 & The application must log the health status of their child, according to section \ref{sec:trafficlight}. \\ \\ 
		GHR3 & The application should log pollen casts for the area the child is in, and which medications were taken each day.\\ \\ 
		GHR4 & The application must store medical plans for their child. These plans concern asthma medications, and contains which medicines should be taken at which times.\\ \\ 
		GHR5 & The application must provide instructions on how to use different medications. These instructions may be pictures or text, provided by NAAF. \\ \\ \hline \\

		CHR1 & The application should distract the children during a treatment. \\ \\ 
		CHR2 & The application should gamify their experience with medication.  \\ \\ \hline \\

		KHR1 & The application should alert children and parent(s) when it is time for a medication/treatment for the children.  \\ \\ 
		KHR2 & The application should distract the children during a treatment. \\ \\ 
		KHR3 & The application should encourage children to take medication through interactivity and gamification.  \\
		\hline
	\end{tabular}
	\caption[High level functional requirements]{High level functional requirements. GHR: GAPP requirement, CHR: CAPP requirement, KHR: karotz application requirement}
	\label{tab:highfuncitionalrequirements}
\end{table}
\clearpage{}

\subsection{Measurement of project effort}
The customer was seeking a documented prototype of a system which could be used for future 
development and for getting additional funding for further development of the project. The customer wanted
the prototype tested on children suffering from diseases causing breathing problems and their parents, in order
to determine whether or not such a system was an adequate solution to the problem. The system should be
compatible with Android v4.0 or newer versions, and should be intuitive to use. The resulting prototype should be well documented
to ensure that further development would be able to continue development after the end of the project.

\subsection{General terms}
We were to make two applications for Android devices and one for Karotz. Originally the customer wanted the smart phone applications made for iOS, 
but since we did not have Apple computers and iPhones, and the customer did not have the funding to provide them, we 
switched to the Android platform. We had at our disposal a Karotz robot with a yellow and a green Nanoz controller,
a Github repository, an AgileZen board and could request an Android tablet to be used for testing if necessary.

\subsection{Planned effort}
The course description states an expected effort of 25 hours per week per student. The course lasted for 13 weeks,
resulting in a total expected effort of 325 hours per student, at a total of 1625 for the team altogether.

\subsection{Schedule of results}
The applications were scheduled to be completed within November 10th. The time we had left after this, was used for fixing
critical errors, and completing the report.

\subsection{Report Outline}
The report is outlined in Table \ref{tab:chapters}. 

\begin{table}
	\begin{center}
	\begin{tabular}{|p{0.3\linewidth}|p{0.6\linewidth}|}   
		\hline   
		\bf{Chapter} & \bf{Description} \\ 
		\hline
			Chapter \ref{chap:intro}: \nameref{chap:intro} &
			 Contains a short description of the project, its goals and purpose and what the 
			 report consists of. \\
		\hline
			Chapter \ref{chap:projectManagement}: \nameref{chap:projectManagement} & 
			Contains a description of how the group is organized and what responsibilities 
			lies on each of the group members. A risk analysis for the project is also included
			 in this chapter, as well as the work plan which describes the different phases, 
			 activities and a Gantt Diagram for the project. Quality Assurance
			 techniques are also discussed, which include meetings, coding templates and document templates. \\
		\hline
			Chapter \ref{chap:prestudy}: \nameref{chap:prestudy} & 
			Contains a documentation of the preliminary studies done ahead of the implementation of the 
			applications, including a report of the design workshop done early in development,
			development methodology, frameworks and tools used in the project, the Karotz platform 
			and information about asthma. \\
		\hline
			Chapter \ref{chap:developmentMethodology}: \nameref{chap:developmentMethodology} &
			Contains description and discussion about the various development methods considered
			for the project, and an analysis of SCRUM, the chosen methodology. \\
		\hline
			Chapter \ref{chap:requrementSpecifications}: \nameref{chap:requrementSpecifications} &
			Contains an overview of the requirement specifications for the system, through use cases
			and functional requirements. \\
		\hline
			Chapter \ref{chap:systemDesign}: \nameref{chap:systemDesign} & 
			Contains a collection of requirements and design choices for the prototype, 
			including use cases, architectural description and documentation of the database. \\
		\hline
			Chapter \ref{chap:testPlan}: \nameref{chap:testPlan} & 
			Describes how the team will do testing throughout the project. \\
		\hline
			Chapter \ref{chap:sprint1}-\ref{chap:sprint5}: \nameref{chap:sprint1}-\nameref{chap:sprint5} & 
			Contains the goals, backlogs, test tables, results and reviews for the respective sprints. \\
		\hline
			Chapter \ref{sec:Usabilitytesting}: \nameref{sec:Usabilitytesting} &
			Contains a description of usability testing in general, and reports and discussion from the 
			usability tests done in the project.\\
		\hline
			Chapter \ref{chap:furtherWork}: \nameref{chap:furtherWork} & 
			Contains a description of what has been implemented and what the next logical steps 
			are based on the current state of the system. \\
		\hline
			Chapter \ref{chap:evaluation}: \nameref{chap:evaluation} & 
			Contains the evaluation and description of how the project was executed. \\
		\hline
	\end{tabular}
	\end{center}
	\caption{Chapters and their respective description}
	\label{tab:chapters}
\end{table}
