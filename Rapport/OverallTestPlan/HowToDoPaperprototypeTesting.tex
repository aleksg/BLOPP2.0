\section{How to do a Usability Test}
\emph{TODO:} Add a bibliography reference for the Apple thing
The following 10 paragraphs for usability testing was originally developed by to psychologists working for Apple: K. Gommoll and A. Nicole. They have later been described by Tognazzini (Tognazzini, 1991).
The 10 paragraphs has later been regarded as a standard for how to do usability testing.
\begin{enumerate}
	\item Introduce yourself and your usability testing team
	\item Explain the reason of the test: to discover errors and problems with an early design of a system.
	\item Tell the participants that they may quit whenever they want. They do not need to explain the reason for quitting.
	\item Explain what equipment is in use. Show the test person the paperprototype and tell them what it should imagine. Tell the test person that the system will work by papers being changed out as the person uses the system, accordingly to what the system is supposed to do. Since the ``system'' is made out of paper, not all functions are possible to animate. It will also take longer time to change the user interface frames, than on a usual computer program.
	\item Teach the test person how to think out loud. Tell the test person it is very important that they speak out what they are doing and what they are thinking while testing, so the observer may understand any difficulties with the prototype. 
	\item Explain why you can't help the test person. The goal of the test is to get the users opinion on the user experience. The test person may ask questions before and after the test.
	\item Describe the tasks and introduce the product. Tell the test person the outline of the test. Give the test person a list of the different tasks he/she will complete during the test. 
	\item Ask if the test person has any questions before starting, then start the test. Take notes while the person is completing the tasks. Also write down any questions you may have.
	\item Complete the test by asking any questions that may have occurred under the test. 
	\item Use the results as input for further work with the design.
\end{enumerate}