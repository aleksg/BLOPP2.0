\chapter{Overall Test Plan}
\label{chap:testPlan}
The following chapter concerns the overall testplan for the project, and it contains general
information about testing and what tests we aimed to perform. The template for our 
tests can be found in table \ref{tab:testtable}, while the tests that were performed can be found
in table \ref{tab:listoftests}.
\\
%\section{Introduction} Almost the same as the section above, so I merge them
We test the system to 
make sure that our applications are easy to use, and works the way they're intended,
ensuring that the delivered product fulfills the requirement specifications. We 
will document the most important tests with a description of the test we performed, what we 
discovered during the test, and what we did to fix eventual problems that surfaced 
during the test.
%Needs references
%rly?

\section{Test methods}
There are different test methods that can be used. The outer points are black-box and 
white-box testing, but we can also do gray-box testing, a combination of the two.

\subsection{Black-box testing}
This method tests the functionality of the system. Black-box testing means we feed 
something in and then see if the result we get out is the same as the one we were 
expecting. This means that knowledge about the code and structure of the system 
we are testing is unknown and irrelevant to the test. The tests will then be based on 
external software descriptions like the functional requirements for the system.

\subsection{White-box testing}
This method of testing concerns itself with testing the internal structures of a 
system. This means knowledge about structure and code is required to run the test, 
and we require knowledge about the programming to design the test cases. Normally 
this type of testing is performed at unit level, but can also be used on the system 
as a whole. The problems detected by white box testing is technical, and it can not 
detect whether or not a program is fulfilling it's functional requirements.

\section{Test levels}
There are different types and levels of testing. Usability testing focuses on the 
general, graphical and functional part of the system, how easy the applications is 
to use for the typical end users. There are low level tests testing the smaller parts 
of the system, typically classes or methods, and there are high level tests, that 
tests the system, or parts of the system.

\subsection{Unit testing}
The lowest form of testing is unit testing. This is intended to test the smallest 
units of the system, namely the methods, classes and variables, making sure that each 
of these parts works as expected. 

\subsection{Module testing}
Once the smaller parts of the system has been tested, we also test the coordination 
between these parts, by testing that the entire module works as we intended.

\subsection{Integration and System testing}
When all of the individually tested modules works as intended, we test 
that the different parts of the system works together, taking 
small to medium parts of the system into the test, while the system tests will test the 
system as a whole. That means that these tests test communication between
the different modules and interfaces.

\section{Testing approach}	
We decided to use both black box and white box testing for our system. We did 
two bigger usability tests, where we applied black box testing on both. We would have preferred to
do some more usability testing, however, since we had to implement the system from scratch, we decided
to start with one paper prototype usability test to get initial thoughts from the potential users. Then 
we moved on with implementing the system before we again did a bigger usability test, this 
time  with children. We used the implemented applications and the distraction sequence with the Karotz.
This meant we had to do alot of unit and integration testing on the system during the implementation. 

All of our tests will follow the template presented in table \ref{tab:testtable}.
\begin{table}
	\begin{center}
		\begin{tabular}{|p{4.0cm}|p{8.0cm}|}
			\hline
			\bf{Item} & \bf{Description}\\
			\hline
			\bf{ID} & Identifier for the test\\
			\bf{Description} & Description of the test\\
			\bf{Date} & Date of the test\\
			\bf{Responsible} & The person responsible for the test\\
			\bf{Subject} & The subject being tested (typically a unit or part of the system)\\
			\bf{Precondition} & The conditions we assume to be in place when the test is started\\
			\bf{Steps} & The steps to perform\\
			\hline
			\bf{Results} & Results after the test was performed\\
			\hline
		\end{tabular}
	\end{center}
	\caption{Test template}
	\label{tab:testtable}
\end{table}

The tests done during this project is listed in table \ref{tab:listoftests}, with the testID, brief description of the test
and during what part of the project it was done.
\begin{table}
	\begin{center}
		\begin{tabular}{|p{3.3cm}|p{10.0cm}|p{4.0cm}|}
			%\endfirsthead
			%\endhead
			%\endfoot
			%\caption{List of tests}
			%\endlastfoot
			\hline
				\bf{ID} & \bf{Description}& \bf{Time of test}\\
			\hline
				USABILITY0.1 &	 Paper prototype test &  						04.09.12 (section \ref{sec:paperprototypetest})\\
				\hline
				USABILITY02 & 	Usability testing of the system on children & 			30.10.12 (table \ref{tab:usability5.1})\\
				\hline
				UNIT1.1 &		Test of GUI for GAPP & 												17.09.12 (table \ref{tab:unit1.1})\\
				\hline
				UNIT1.2 & 		Test of GUI for CAPP & 												17.09.12 (table \ref{tab:unit1.2})\\
				\hline
				UNIT2.1 & 		Test of the CAPP distraction sequence & 									30.09.12 (table \ref{tab:unit2.1})\\
				\hline
				UNIT2.2 & 		Test of the database connection & 										26.09.12 (table \ref{tab:unit2.2})\\
				\hline
				UNIT2.3 & 		Testing of SQL-queries &		 										30.09.12 (table \ref{tab:unit2.3})\\
				\hline
				UNIT3.1 & 		Test that alarm is given independently of phone state & 							14.10.12 (table \ref{tab:unit3.1})\\
				\hline
				UNIT3.2 & 		Test that the correct days is colored in the log &								09.10.12 (table \ref{tab:unit3.2})\\
				\hline
				UNIT3.3 & 		Test that the karotz notification is given at the correct time & 						09.10.12 (table \ref{tab:unit3.3})\\
				\hline
				UNIT3.4 & 		Test that the karotz distraction runs after recieving the notification and starting the sequence & 	09.10.12 (table \ref{tab:unit3.4})\\
				\hline
				UNIT3.5 & 		Test that the notifications with multiple doses makes the correct amount of medications & 		11.10.12 (table \ref{tab:unit3.5})\\
				\hline
				UNIT4.1 & 		Test that the right instructions are downloaded and shown on the instructions menu screen & 	18.10.12 (table \ref{tab:unit4.1})\\
				\hline
				UNIT5.1 & 		Test of the web access module \code{add\_child.php} & 							30.10.12 (table \ref{tab:unit5.1})\\
				\hline
				UNIT5.2 & 		Test of the web access module \code{add\_plan\_dose.php} & 						01.11.12 (table \ref{tab:unit5.2})\\
				\hline
				UNIT5.3 & 		Test of the web access module \code{dose\_is\_taken.php}. & 						01.11.12 (table \ref{tab:unit5.3})\\
				\hline
				UNIT5.4 & 		Test of the web access module \code{get\_available\_child\_states.php}. & 				01.11.12 (table \ref{tab:unit5.4})\\
				\hline
				UNIT5.5 & 		Test of the web access module \code{get\_child.php}. & 							01.11.12 (table \ref{tab:unit5.5})\\
				\hline
				UNIT5.6 & 		Test of the web access module \code{get\_child\_state.php}. & 						01.11.12 (table \ref{tab:unit5.6})\\
				\hline
				UNIT5.7 & 		Test of the web access module \code{get\_doses\_for\_current\_state.php}. & 			01.11.12 (table \ref{tab:unit5.7})\\
				\hline
				UNIT5.8 & 		Test of the web access module \code{get\_instructions.php}. & 						01.11.12 (table \ref{tab:unit5.8})\\
				\hline
				UNIT5.9 & 		Test of the web access module \code{get\_log\_days\_for\_child.php}. & 				03.11.12 (table \ref{tab:unit5.9})\\
				\hline
				UNIT5.10 & 		Test of the web access module \code{get\_log\_for\_child.php}. & 					03.11.12 (table \ref{tab:unit5.10})\\
				\hline
				UNIT5.11 & 		Test of the web access module \code{get\_plan.php}. & 							04.11.12 (table \ref{tab:unit5.11})\\
				\hline
				UNIT5.12 & 		Test of the web access module \code{register\_medicine\_taken.php}. & 				04.11.12 (table \ref{tab:unit5.12})\\
				\hline
				UNIT5.13 & 		Test of the web access module \code{remove\_plan\_dose.php}. & 					04.11.12 (table \ref{tab:unit5.13})\\
				\hline
				UNIT5.14 & 		Test of the web access module \code{remove\_plan\_medicine\_at\_time.php}. & 			04.11.12 (table \ref{tab:unit5.14})\\
				\hline
				UNIT5.15 & 		Test of the web access module \code{set\_child\_state.php}. & 						05.11.12 (table \ref{tab:unit5.15})\\
				\hline
				INTEGRATION5.1 & Test of CAPPs alarm and distraction sequences. & 							06.11.12 (table \ref{tab:integration5.1})\\
				\hline
				INTEGRATION5.2 & Testing that the log updates correctly based on registered medication and pollen feed. & 	06.11.12 (table \ref{tab:integration5.2})\\
				\hline
				INTEGRATION5.3 & Testing that the medicationplans is correctly registered to their respective healthstates. & 	05.11.12 (table \ref{tab:integration5.3})\\
			\hline
		\end{tabular}
	\end{center}
	\caption{List of tests}
	\label{tab:listoftests}
\end{table}