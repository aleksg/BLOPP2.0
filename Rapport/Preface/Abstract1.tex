\section*{Abstract}
\setcounter{page}{1}
\pagenumbering{roman}
%This master-level course TDT4290 Customer Driven Project deals with a project assignment that is mandatory for all computer science students. 
%The purpose of this project is to teach students, by working in groups, software engineering skills in context of a development project. 
%The students are to make a realistic prototype of an Information System “on contract” for a real world customer.
% Hæ?

%Our Customer is NAAF (Norges Astma og Allergiforbund) and ``Sykehusapotekene i Norge''.  Contact information for both the customer and 
%the development team is listed in the Introduction Chapter.  

%The development team consists of Esben, Aleksander, Jørgen, Yngve and Eirik, all Computer Science students at NTNU.
% The students are taking this course as a part of their master degree in computer science.
% In very brief summary, our assignment and project goal was to create two applications in order to motivate both children and parents to 
% follow up a medicational plan. For the application made for children the purpose is to make a better and more fun way for the children 
% to take asthma-medication with use of mobile technology. For the application made for parents the purpose is to make it easier to learn 
% how to take medicine the right way, and follow up and log a medicational plan. 

This project aimed to create three applications to motivate and remind asthmatic children to 
take their medication. When children are on a medication plan, taking the medicine might be boring or stressful 
because they are reminded of their asthma, are disturbed in their routine, or that the medication
process itself is scary. The use of an appealing figure like the rabbit robot Karotz provides a way to avoid
some of these concerns. In combination with a reminder and distraction application and an adult information
and settings application, the complete system could help leviate the burden of medicating asthmatic children.

The three applications were developed on two different platforms. The guardian application for configuration, 
teaching and viewing a log, and a children application for teaching, reminding and distracting during treatment were developed 
for the Android platform. A second application for reminding and distracting the children during treatment was made for 
Karotz. Through the agile software development technique SCRUM, the project team completed five sprints of 
iterative study, planning, programming, adaption and testing. The Android applications are written in Java, 
while the Karotz application is written in JavaScript. A central database is written in MySQL 
with PHP sites for access through the internet protocol HTTP.

The prototype system is developed for Sykehusapotekene i Midt-Norge as a part of NTNU's course TDT4290 --- Customer Driven Project.

\paragraph{Keywords:}
\emph{BLOPP, Asthma, Gamification, Android, Karotz, SCRUM, Software development}
% dette er en fin abstract, men den er litt kort.
% TODO: beskrive litt mer rundt prosjektet: platformene (android + karotz), NTNU-faget (bittelitt på slutten, ingen navn)

\vfill
\noindent \signbox{J{\o}rgen Aaberg} \hfill \signbox{Esben Aarseth}
\signbox{Eirik Skjeggestad Dale} \hfill \signbox{Aleksander Gisvold}
\begin{center}\signbox{Yngve Svalestuen}\end{center}
\vspace*{4cm}