\subsection{Sprints}
\label{sec:sprints}
This section gives a short description of how the Scrum development method was used in the project. 
For a general explanation of Scrum see Section \ref{sec:scrum}.

\subsubsection{Sprint duration}
We decided on having 14 day sprints. After discussions with the customer,
we agreed that this would be a suitable duration, due to the fact 
that the documentation needed for each sprint would be time consuming 
for shorter sprints.

\subsubsection{Sprint Planning Meeting}
To start each sprint, we held a sprint planning meeting. During this meeting, we discussed which user 
stories/epics from the sprint backlog should be worked on during the sprint. The reason behind such a 
meeting is to make sure the team is on updated on the goals for the following sprint. To decide what 
user stories/epics should be chosen, the priorities given by the customer was used as a pinpoint. If 
the customer wanted to make any changes during a sprint, the changes were noted and discussed during 
the next sprint planning meeting.

\subsubsection{Daily Standup}
The daily standups (also commonly known as daily scrum meeting) were held on Mondays, Wednesdays and 
Fridays. The team decided on this semi-daily recurrence since not all team members were able to work 
on the project every day. During the standup meetings all team members would answer three questions: 
what have you done since our last meeting, what will you work on until the next meeting, and what 
problems did occur since our last meeting?

Answering these questions gave a certain status update, and made it easier to re-assign team members 
to tasks if needed. During the standups all technical discussions were discouraged. If any technical 
questions arose, the people involved would discuss this after the meeting, to make sure they were not 
wasting other people's time. Each standup had a max allowed length of 15 minutes. 

\subsubsection{Sprint retrospective}
The sprint retrospective is the written conclusion of the sprint. A meeting was held at the end of each 
sprint, discussing the results throughout the sprint, both finished and unfinished tasks. The tasks not 
completed were moved to the next sprint, and the reason for the task not being completed was stated in 
the sprint report. 

The sprint report also includes an update of the sprint backlog, along with an overview of how much time 
was spent on each task, making it easy to compare to the time estimate. 

The sprint retrospective also contains a burndown chart, giving a visual representation of how the team 
worked during the sprint.

For each sprint we answered the following questions: 
\begin{itemize}
	\item What went well?
	\item What shall we start doing?
	\item What could have gone better?
	\item What should we stop doing?
\end{itemize}
%TODO: the questions

\subsubsection{Explanation of Sprint Backlog}
The sprint backlog is a task management tool to document and ensure the progress of the sprint. Each task the 
team chooses to focus on in the sprint is enlisted. The task is given an ID and already has a name. The 
Function number is an hour-independent number telling how difficult the team expects the task to be. The 
base number represents how many hours the team expects to work to finish one story point. The base number multiplied 
with the function number for a task gives the estimated work hours needed to finish a task.

The base number may change from sprint to sprint, but not during a sprint. The team did an evaluation of 
the base number in advance of each sprint, to make good estimates.

The name column is used to keep track of who is responsible for the task. This may change during the sprint, 
but the sprint backlog should always be showing the correct info.

Based on how many team members are available and how many work hours they may put in, the team gives an 
expected decrease of the story points left. This is reflected in the sprint burndown chart for each sprint, 
as a straight decreasing line.