\section{Privacy and security}
Early in the project we did some research into what information we could be legally allowed to keep track off, and 
more importantly, what was recommended to avoid doing, to be able to avoid issues later. Specifically we looked into
what information we could save regarding how well the medicationplans were followed, and in if this information could
be sent to medical staff, with information about who followed the plans well and who did not. We discovered it was not
necessarily legal to send this information without consent from the person the data was collected from, or in our case, 
the guardian of that person. We thought about making this information 
available to medical staff, but dropped it since it is illegal to send
this type of information without consent.

In terms of whether or not we could use the child's personal number as an identifier, we discovered that this was legal, but only 
if we had an actual need to save it, that we had pursuant in the law to save it, and finally, that satisfactory identification 
of the relevant person could not be achieved in any other way. For the system we were making we had none of these 
criteria in order, but for future expansions this might be information actually required. Hospitals have to be sure they 
give medicines to the correct people, so the personal number is already written on any prescription medicine used. If the 
system is expanded towards the hospital nebulization treatment it might be necessary or useful to store the personal number
in our database, to connect the user with the person receiving the treatment.