\subsection{Traffic Light Classification of Asthma Condition}
\label{sec:trafficlight}

The traffic light program is a way of classifying condition and resulting medication for asthmatic 
patients. It is a very simple system that requires very little knowledge to understand, and is 
therefore well-suited for a project aimed at children. The basic outline can be compared to a traffic
light.

Borge et al. (2002)\cite{livingwithasthma} defines the zones as \emph{green}, \emph{orange}, and \emph{red}
and describes effects and treatment for each of them.

\subsubsection{Green Zone}
Green is the normal zone. A patient in the green zone can be described as in ``regular condition''. 
He or she is breathing normally, even when doing light physical exercise.

When an asthmatic is the green zone, it is normal to take two to three different medicines each day,
often with cortisone

\subsubsection{Yellow Zone}
Yellow is the ``ill'' zone. When a patient is in the yellow zone, they exhibit moderate signs of illness
such as breathlessness and coughing. There may also be allergy reactions, and waking up at night from
breathlessness and coughing. A patient may be defined as being in the yellow zone if he or she has a cold.

In the yellow zone, patients typically take more medication than in the green zone. A normal amount
is 4 to 6 doses daily. The medication from the green plan is taken as normal, in addition to any new
medicines introduced by the yellow state.

\subsubsection{Red Zone}
The red zone is labeled the ``stop'' zone. A patient in the red zone will have almost closed airways,
making it very difficult to breathe and the person will have to stop any activities.

A patient in the red zone has to fix his or her state immediately. This could be by opening windows,
finding a good resting position, taking specific emergency medication or using specific breathing
techniques. If these courses of action don't help, the patient should immediately call the doctor.