\section{Functional Requirements}
\label{sec:functionalRequirements}
As we are developing three separate applications, we found it best to consider
the functional requirements separately. 
We will use the abbreviations PFR (Parent Functional Requirement), 
CFR (Child Functional Requirement) and KFR (Karotz Functional Requirement).

\subsection{GAPP - Guardian Application}

\subsubsection{PFR 1 - Medication plan}
The application should make it easier for parents and responsible adults to keep
track of the medication plan of their children.
\\
Priority: High

\subsubsection{PFR 2 - Notifications}
When a child needs to take a preventive medication, the application should
send a notification to remind the parents or other responsible adults,
either through the smartphone/tablet, through Email, or via Karotz.
\\
Priority: High

\subsubsection{PFR 2.1 - Settings for notifications}
The user should be able to choose settings for notifications. Example for
settings to be made are: Time for notification, Notification appearance, and so
on.
\\
Priority: Medium

\subsubsection{PFR 2.2 - Notification to change condition}
The user should get a weekly reminder to update the condition of the child, 
if you're outside the default medicationplan.
\\
Priority: Medium

\subsubsection{PFR 3 - Families}
The application should support several children.
\\
Priority: Low

\subsubsection{PFR 4 - Guidelines}
The application should provide guidelines for how to use a medicine correctly,
and how to do a treatment correctly.
\\
Priority: High

\subsubsection{PFR 4.1 - Guidelines from NAAF}
The application should provide support for changeable guidelines from NAAF in
form of series of pictures, text or animations/movies.
\\
Priority: Medium

\subsubsection{PFR 5 - Keep records of condition}
The application should be able to keep track of the child's previous medical
condition (in terms of days). The condition is to be set by a user, and varies
from Green, Orange and Red zone as described in Appendix \ref{sec:trafficlight}.
\\
Priority: High

\subsubsection{PFR 6 - Pollen forecast}
The application should be able to use pollen forecasts to warn parents about
possible bad days. The forecast should be included in the log.
\\
Priority: Medium

\subsubsection{PFR 7 - Screen sizes}
The application should have support for different screen-sizes.
\\
Priority: Low

\subsection{CAPP - Children's Application}


\subsubsection{CFR 1 - Distraction}
The user should be able to choose a distraction during medication. This could be
via an external application, a video or one of the Karotz in the household.
\\
Priority: High

\subsubsection{CFR 2 - Rewards}
When a child is done with a treatment, he/she should get a reward.
\\
Priority: High

\subsubsection{CFR 2.1 - Rewards}
The child should feel that the reward gives something back, for instance be able to buy something from
an avatar-shop, etc. 
\\
Priority: Low


\subsubsection{CFR 3 - Screen sizes}
The application should have support for different screen-sizes.
\\
Priority: Low

\subsubsection{CFR 4 - Avatar}
The application should have an avatar for each child, that can be chosen by the child, and be
customized through a shop, providing different clothes and items a child could collect. The shop should use the rewards as currency.
\\
Priority: Low

\subsubsection{CFR 5 - Child friendly instructions}
The application should give instructions to a child, that a child can easily comprehend. Both as part of the distraction, 
and as a separate part of the application.  
\\
Priority: High

\subsection{Karotz Application}

\subsubsection{KFR1 - Notification}
The application should make children aware of medication-time by playing sound,
movement of ears and through the color of the Karotz. 
\\
Priority: High
\subsubsection{KFR2 - Distraction}
The application should be able to distract children when taking medication. This
should be through movement of ears and ability to play some sound (music,
book reading, etc.)
\\
Priority: High
\subsubsection{KFR3 - Reward}
The application should reward children by telling how many stars the child earned. The stars should be stored in the database and sync with the application.
\\
Priority: High
\subsubsection{KFR4 - Register use of medicine}
The application should allow the child to register when it is done taking the medicine, by
holding a RFID-chip close to the karotz.
\\
Priority: Medium

\subsubsection{KFR5 - Logging}
The application should be able to save the health status to the database
according to the child's health status.
\\ 
Priority: Medium


