\section{Sprint }


The following section presents an overview of how we planned, worked and
completed sprint .

\\
Sprint X started on XXth of  and ended on XXth of , giving it a duration of 14 days. The length of the sprints were discussed with the customer and the advisor, and all agreed on 14 days as a suitable length of a sprint.

\\
The chapter is divided into five parts, starting with the overall plan for the
sprint in Section \ref{sec:sprintplan}. Followed by the sprint backlog, which
enlists the tasks that have been chosen for the sprint. Section
\ref{sec:designAndImplementation}.
will focus on the work made to the GUI, the logic implemented in the application and the work done to the database and database access in the application.
The chapter ends with what have been tested and the corresponding results in
Section \ref{sec:testingAndResults} and a sprint review in Section
\ref{sec:sprintRetrospective}.

\subsection{Sprint Plan}
\label{sec:sprintplan}



\subsection{Sprint backlog}
This section contains a table with the sprint backlog, which is a smaller part of the product backlog. The goal is to get the entire backlog implemented during the sprints.

\begin{table}[h]
\begin{center}
\begin{tabular}{|p{2.0cm}| p{5.0cm}| p{2.0cm}|p{2.0cm}|p{2.0cm}|}
\hline
\#  ID 	& Task 	& Story points 	& Estimated hours & Responsible \\
\hline
1.1 & & & & \\
\hline
1.2 & & & & \\
\hline
1.3 & & &  & \\
\hline
1.4 & & & & \\
\hline
2.1 & & &  & \\
\hline
2.2 & & & & \\
\hline
3.1 & & & & \\
\hline
3.2 & & & & \\
\hline
3.3 & & & & \\
\hline
3.4 & &  & &  \\
\hline
\end{tabular}
\end{center}
\end{table}





\subsection{Design and Implementation}
\label{sec:designAndImplementation}


\subsubsection{User Interface Layer}



\subsubsection{Application Logic Layer}



\subsubsection{Data Persistence Layer}



\subsection{Testing and Results}
\label{sec:testingAndResults}

\subsubsection{Testing}


\newpage{}
\begin{table}[h]
\begin{center}
\begin{tabular}{|p{4.0cm}|p{8.0cm}|}
\hline
\bf{Item} & \bf{Description}\\
\hline
\bf{ID} & \\
\bf{Description} & \\
\bf{Date} & \\
\bf{Responsible} & \\
\bf{Subject} & \\
\bf{Precondition} & \\
\bf{Steps} &
\begin{enumerate}

\end{enumerate}\\
\bf{Results} & \\
\hline
\end{tabular}
\end{center}
\end{table}
\newpage{}
\begin{table}[h]
\begin{center}
\begin{tabular}{|p{4.0cm}|p{8.0cm}|}
\hline
\bf{Item} & \bf{Description}\\
\hline
\bf{ID} & \\
\bf{Description} & \\
\bf{Date} & \\
\bf{Responsible} & \\
\bf{Subject} & \\
\bf{Precondition} & \\
\bf{Steps} & \\
\bf{Results} & 
\begin{enumerate}

\end{enumerate}\\
\hline
\end{tabular}
\end{center}
\end{table}

\begin{table}[h]
\begin{center}
\begin{tabular}{|p{4.0cm}|p{8.0cm}|}
\hline
\bf{Item} & \bf{Description}\\
\hline
\bf{ID} & \\
\bf{Description} & \\
\bf{Date} & \\
\bf{Responsible} & \\
\bf{Subject} & \\
\bf{Precondition} & \\
\bf{Steps} & \\ 
\bf{Results} & 
\\
\hline
\end{tabular}
\end{center}
\end{table}

\subsubsection{Results}


\subsection{Sprint Retrospective}
\label{sec:sprintRetrospective}


\paragraph{What went well?}


\paragraph{What shall we start doing?}
\begin{enumerate}

\end{enumerate}

\paragraph{What could have gone better?}



\paragraph{What should we stop doing?}


\subsubsection{Sprint Burndown Chart}
\newpage{}
\begin{table}[h]
\begin{sideways}
\begin{tabular}{l p{6.5cm} l l l l l }
\hline
  \#  ID 	& Task 	& Story points 	& Estimated & Actual &
  Estimated Left & Responsible \\
\hline
  1.1 & & & &  &  & \\
\hline
  1.2 & & & & & & \\
\hline
  1.3 & & & & & & \\
\hline
  1.4 & & & & & & \\
\hline
  2.1 & & & & & & \\
\hline
2.2 & & & & & & \\
\hline
3.1 & & & & & & \\
\hline
3.2 & & & & & & \\
\hline
4.1 & & & & & & \\
\hline
4.2 & & & & & &  \\
\hline
4.3 & & & & & & \\
\hline
4.4 &  & & & & & \\ 
\hline 
5.1 & & & & & &  \\ 
\hline 
5.2 & & & & & &  \\ 
\hline 
\end{tabular}
\end{sideways}
\end{table}