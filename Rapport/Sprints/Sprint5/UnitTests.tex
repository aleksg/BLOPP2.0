Tables \ref{tab:unit5.1} through \ref{tab:unit5.15} describe the unit tests done in sprint 5. 

\begin{table} %UNIT 5.1: add_child
	\begin{center}
		\begin{tabular}{|p{3.0cm}|p{14.0cm}|}
			\hline
			\bf{Item} & \bf{Description}\\
			\hline
			\bf{ID} & UNIT5.1\\
			\bf{Description} & Test of the web access module \code{add\_child.php} \\
			\bf{Date} & 30.10.12\\
			\bf{Responsible} & Yngve\\
			\bf{Subject} & \code{add\_child.php} and database \\
			\bf{Precondition} & The database is up with the tables \code{MEDICAL\_PLANS}, \code{CHILDREN}and \code{HEALTH\_STATES}.\\
			\bf{Steps} & 
			    \begin{tabulenum}
			        \item Initiate a REST client (POSTMAN) with the URL \url{http://folk.ntnu.no/yngvesva/blopp/add\_child.php}.
			        \item Add the POST data
			        \begin{tabulitem}
			            \item \code{name}: testname
			            \item \code{persnum}: 10101012345
			            \item \code{states[]}: 1
			            \item \code{states[]}: 2
			        \end{tabulitem}
			        \item Observe the returned JSON data
			    \end{tabulenum}\\
		    \hline
			\bf{Results} & 
				The JSON data returned was:
\begin{lstlisting}[caption=JSON result from \code{add\_child.php}]
{
	post: {
		name: "testnavn",
		persnum: "10101012345",
		states: [
			"1",
			"2"
		]
	},
	q: "INSERT INTO `CHILDREN` (`id`, `name`, `pers_num`, `medical_plan_id`, `credits`) VALUES ('', 'testnavn', '10101012345', '0', '0')",
	medical_plan_q: "INSERT INTO `MEDICAL_PLANS` (`id`, `label`) VALUES ('', 'testnavn')",
	medical_plan_id: 0,
	child_id: 12,
	state_queries: [
		"INSERT INTO `CHILD_HEALTH_STATES` (`child_id`, `health_state_id`, `applies_now`, `default`) VALUES ('12', '1', '1''1')",
		"INSERT INTO `CHILD_HEALTH_STATES` (`child_id`, `health_state_id`, `applies_now`, `default`) VALUES ('12', '2', '0''0')"
	],
	all_state_ids: [
		"1",
		"2",
		"3"
	]
}
\end{lstlisting}
				We can see from the JSON result that a child was added successfully, and it got the ID 10.
				\\
			\hline
		\end{tabular}
	\end{center}
	\caption{Unit test 5.1, \code{add\_child.php}}
	\label{tab:unit5.1}
\end{table}

\begin{table} %UNIT 5.2: add_plan_dose
	\begin{center}
		\begin{tabular}{|p{3.0cm}|p{14.0cm}|}
			\hline
			\bf{Item} & \bf{Description}\\
			\hline
			\bf{ID} & UNIT5.2\\
			\bf{Description} & Test of the web access module \code{add\_plan\_dose.php}\\
			\bf{Date} & 01.11.12\\
			\bf{Responsible} & Yngve\\
			\bf{Subject} & \code{add\_plan\_dose.php} and the database\\
			\bf{Precondition} & The database is up with the tables \code{MEDICAL\_PLAN\_DOSES} and \code{CHILDREN}.\\
			\bf{Steps} &
			\begin{tabulenum}
				\item Initiate a REST client (POSTMAN) with the url \url{http://folk.ntnu.no/yngvesva/blopp/add\_plan\_dose.php}.
				\item Add the post data:
					\begin{tabulitem}
					  \item \code{child\_id}: 6
					  \item \code{health\_state\_id}: 1
					  \item \code{medicine\_id}: 1
					  \item \code{time}: 12:34:56
					\end{tabulitem}
				\item Observe the returned JSON data.
			\end{tabulenum}\\
			\hline
			\bf{Results} & The returned JSON data was:
\begin{lstlisting}[caption=JSON result from \code{add\_plan\_dose.php}]
{
    "sqlsuccess": true,
    "q": "INSERT INTO `MEDICAL_PLAN_DOSES` (`id`, `medical_plan_id`, `health_state_id`, `time`, `medicine_id`) SELECT '', C.medical_plan_id, '1', '12:34:56', '1' FROM `CHILDREN` AS C WHERE C.id='6'",
    "child_id": "6",
    "health_state_id": "1",
    "medicine_id": "1",
    "time": "12:34:56",
    "id": 40
}
\end{lstlisting}
				We see that a planned dose was added successfully by validating the queries and the parameter \code{sqlsuccess}, and checking the id 40.\\
			\hline
		\end{tabular}
	\end{center}
	\label{tab:unit5.2}
	\caption{Unit test 5.2, \code{add\_plan\_dose.php}}
\end{table}

\begin{table} %UNIT 5.3: dose_is_taken
	\begin{center}
		\begin{tabular}{|p{3.0cm}|p{14.0cm}|}
			\hline
			\bf{Item} & \bf{Description}\\
			\hline
			\bf{ID} & UNIT5.3, \\
			\bf{Description} & Test of the web access module \code{dose\_is\_taken.php}.\\
			\bf{Date} & 01.11.12\\
			\bf{Responsible} & Yngve\\
			\bf{Subject} & The database and \code{dose\_is\_taken.php}.\\
			\bf{Precondition} & The database is up with the table \code{DAY\_MEDICINE\_DOSES}.\\
			\bf{Steps} &
			\begin{tabulenum}
				\item Initiate a web browser with the GET url \url{http://folk.ntnu.no/yngvesva/blopp/dose\_is\_taken.php?dose\_id=40}
				\item Observe the returned JSON data.
			\end{tabulenum}\\
			\hline
			\bf{Results} & The returned JSON data was:
\begin{lstlisting}[caption=JSON result from \code{dose\_is\_taken.php}]
{
	sqlsuccess: true,
	dose_id: "40",
	day_date: "2012-11-01",
	query: "SELECT `id` FROM `DAY_MEDICINE_DOSES` WHERE `medical_plan_dose_id`='40' AND `day_date`='2012-11-01' LIMIT 0,1",
	result: false
}
\end{lstlisting}
			We see from the result that the dose with id 40 (added in UNIT5.2, table 
			\ref{tab:unit5.2}) has not been taken on the actual date, and by the query
			and the \code{sqlsuccess} parameter, we see that the GET module works.\\
			\hline
		\end{tabular}
	\end{center}
	\caption{Unit test 5.3, \code{dose\_is\_taken.php}}
	\label{tab:unit5.3}
\end{table}

\begin{table} %UNIT 5.4: get_available_child_states
	\begin{center}
		\begin{tabular}{|p{3.0cm}|p{14.0cm}|}
			\hline
			\bf{Item} & \bf{Description}\\
			\hline
			\bf{ID} & UNIT5.4\\
			\bf{Description} & Test of the web access module \code{get\_available\_child\_states.php}.\\
			\bf{Date} & 01.11.12\\
			\bf{Responsible} & Yngve\\
			\bf{Subject} & The database and \code{get\_available\_child\_states.php}.\\
			\bf{Precondition} & A working database with the tables \code{HEALTH\_STATES} and \code{CHILD\_HEALTH\_STATES}.\\
			\bf{Steps} &
			\begin{tabulenum}
				\item Initiate a web browser with the GET url \url{http://folk.ntnu.no/yngvesva/blopp/get\_available\_child\_states.php?child\_id=6}.
				\item Observe the returned JSON data.
			\end{tabulenum}\\
			\hline
			\bf{Results} & The returned JSON data was:
\begin{lstlisting}[caption=JSON result from \code{get\_available\_child\_states.php}]
{
	sqlsuccess: true,
	child_id: "6",
	query: "SELECT id, label FROM `HEALTH_STATES` HS WHERE HS.id IN (SELECT health_state_id FROM `CHILD_HEALTH_STATES` CHS WHERE CHS.child_id=6) LIMIT 0,10",
	rows: [
		{
			id: "1",
			label: "GREEN"
		},
		{
			id: "2",
			label: "YELLOW"
		},
		{
			id: "3",
			label: "RED"
		}
	]
}
\end{lstlisting}
			We see from the returned data that the child with $ID=6$ can have all states (\code{GREEN}, 
			\code{YELLOW} and \code{RED}), and that the module works by checking \code{sqlsuccess} and \code{query}.\\
			\hline
		\end{tabular}
	\end{center}
	\caption{Unit test 5.4, \code{get\_available\_child\_states.php}}
	\label{tab:unit5.4}
\end{table}

\begin{table} %UNIT 5.5: get_child
	\begin{center}
		\begin{tabular}{|p{3.0cm}|p{14.0cm}|}
			\hline
			\bf{Item} & \bf{Description}\\
			\hline
			\bf{ID} & UNIT5.5\\
			\bf{Description} & Test of the web access module \code{get\_child.php}.\\
			\bf{Date} & 01.11.12\\
			\bf{Responsible} & Yngve\\
			\bf{Subject} & The database and \code{get\_child.php}.\\
			\bf{Precondition} & A working database with the table \code{CHILDREN}.\\
			\bf{Steps} &
			\begin{tabulenum}
				\item Initiate a web browser with the GET url \url{http://folk.ntnu.no/yngvesva/blopp/get\_child.php?child\_id=6}.
				\item Observe the returned JSON data.
			\end{tabulenum}\\
			\hline
			\bf{Results} &
			The returned JSON data was:
\begin{lstlisting}[caption=JSON result from \code{get\_child.php}]
{
	sqlsuccess: true,
	child_id: "6",
	query: "SELECT * FROM `CHILDREN` WHERE id=6 LIMIT 0,1",
	information: {
		id: "6",
		name: "Hermann",
		pers_num: "1010354322",
		medical_plan_id: "5",
		avatar_id: "7",
		credits: "45",
		location_latitude: "0",
		location_longitude: "0"
	}
}
\end{lstlisting}
			We see from the returned data that the child with $ID=6$ is named Hermann, some other information, and that the query works.
			\\
			\hline
		\end{tabular}
	\end{center}
	\caption{Unit test 5.5, \code{get\_child.php}}
	\label{tab:unit5.5}
\end{table}

\begin{table} %UNIT 5.6: get_child_state
	\begin{center}
		\begin{tabular}{|p{3.0cm}|p{14.0cm}|}
			\hline
			\bf{Item} & \bf{Description}\\
			\hline
			\bf{ID} & UNIT5.6\\
			\bf{Description} & Test of the web access module \code{get\_child\_state.php}\\
			\bf{Date} & 01.11.12\\
			\bf{Responsible} & Yngve\\
			\bf{Subject} & The database and \code{get\_child\_state.php}.\\
			\bf{Precondition} & A working database with the tables \code{HEALTH\_STATES} and \code{CHILD\_HEALTH\_STATES}\\
			\bf{Steps} &
			\begin{tabulenum}
				\item Initiate a web browser with the GET url \url{http://folk.ntnu.no/yngvesva/blopp/get\_child\_state.php?child\_id=6}.
				\item Observe the returned JSON data.
			\end{tabulenum}\\
			\hline
			\bf{Results} & The returned JSON data was:
\begin{lstlisting}[caption=Returned JSON data from \code{get\_child\_state.php}]
{
	sqlsuccess: true,
	child_id: "6",
	query: "SELECT id, label FROM `HEALTH_STATES` HS WHERE HS.id IN (SELECT health_state_id FROM `CHILD_HEALTH_STATES` CHS WHERE CHS.child_id=6 AND CHS.applies_now=1) LIMIT 0,1",
	state: {
		id: "1",
		label: "GREEN"
	}
}
\end{lstlisting}
			We see from the returned data that the child with $ID=6$ is in the green state, and that the module works.\\
			\hline
		\end{tabular}
	\end{center}
	\caption{Unit test 5.6, \code{get\_child\_state.php}}
	\label{tab:unit5.6}
\end{table}

\begin{table} %UNIT 5.7: get_doses_for_current_state
	\begin{center}
		\begin{tabular}{|p{3.0cm}|p{14.0cm}|}
			\hline
			\bf{Item} & \bf{Description}\\
			\hline
			\bf{ID} & UNIT5.7\\
			\bf{Description} & Test of the web access module \code{get\_doses\_for\_current\_state.php}.\\
			\bf{Date} & 01.11.12\\
			\bf{Responsible} & Yngve\\
			\bf{Subject} & The database and \code{get\_doses\_for\_current\_state.php}\\
			\bf{Precondition} & A working database with the tables \code{MEDICAL\_PLAN\_DOSES}, \code{DAY\_MEDICINE\_DOSES} and \code{CHILD\_HEALTH\_STATES}.\\
			\bf{Steps} &
			\begin{tabulenum}
				\item Initiate a web browser with the GET url \url{http://folk.ntnu.no/yngvesva/blopp/get\_doses\_for\_current\_state.php?child\_id=6}.
				\item Observe the returned JSON data.
			\end{tabulenum}\\
			\hline
			\bf{Results} & The returned JSON data was:
\begin{lstlisting}[caption=Returned JSON data for \code{get\_doses\_for\_current\_state.php}]
{
	sqlsuccess: true,
	child_id: "6",
	query: "SELECT Mpd.id, Mpd.medical_plan_id, Mpd.health_state_id, Mpd.time, Mpd.medicine_id, M.color AS 'medicine_color', M.name AS 'medicine_name' FROM `MEDICAL_PLAN_DOSES` AS Mpd, `MEDICINES` AS M WHERE Mpd.medical_plan_id IN (SELECT C.medical_plan_id FROM `CHILDREN` AS C WHERE C.id=6) AND Mpd.id NOT IN (SELECT DMD.medical_plan_dose_id FROM `DAY_MEDICINE_DOSES` AS DMD WHERE DMD.medical_plan_dose_id=Mpd.id AND DMD.day_date='2012-11-04') AND Mpd.medicine_id = M.id AND Mpd.health_state_id IN (SELECT HS.health_state_id FROM `CHILD_HEALTH_STATES` AS HS WHERE HS.child_id=6 AND HS.applies_now='1') LIMIT 0,100",
	rows: [
		{
			id: "41",
			medical_plan_id: "5",
			health_state_id: "1",
			time: "01:09:00",
			medicine_id: "1",
			medicine_color: "BLUE",
			medicine_name: "Flutide"
		}
	]
}
\end{lstlisting}
			From the returned data we see that the child with $ID=6$ has yet to take Flutide today, which 
			should be taken at 01:09:00. We can also validate the SQL query, and see that $sqlsuccess=true$,
			which means that the module works.\\
			\hline
		\end{tabular}
	\end{center}
	\caption{Unit test 5.7, \code{get\_doses\_for\_current\_state.php}}
	\label{tab:unit5.7}
\end{table}

\begin{table} %UNIT 5.8: get_instructions
	\begin{center}
		\begin{tabular}{|p{3.0cm}|p{14.0cm}|}
			\hline
			\bf{Item} & \bf{Description}\\
			\hline
			\bf{ID} & UNIT5.8\\
			\bf{Description} & Test of the web access module \code{get\_instructions.php}.\\
			\bf{Date} & 01.11.12\\
			\bf{Responsible} & Yngve\\
			\bf{Subject} & The database and \code{get\_instructions.php}.\\
			\bf{Precondition} & A working database with the table \code{MEDICINE\_INSTRUCTIONS}.\\
			\bf{Steps} &
			\begin{tabulenum}
				\item Initialize a web browser with the GET url \url{http://folk.ntnu.no/yngvesva/blopp/get\_instructions.php?medicine\_id=1}.
				\item Observe the returned JSON data.
			\end{tabulenum}\\
			\hline
			\bf{Results} & The returned JSON data was: 
\begin{lstlisting}[caption=Returned JSON from \code{get\_instructions.php}]
{
	sqlsuccess: true,
	child_id: "1",
	query: "SELECT * FROM `MEDICINE_INSTRUCTIONS` WHERE id IN (SELECT instructions_id FROM `MEDICINES` WHERE `id`='1') LIMIT 0,1",
	instructions: {
		id: "1",
		url: "medicine_flutide_50microg.jpg",
		information: "Et inhalasjonssteroid som brukes fast slik legen har bestemt. Effekten ses forst etter noen dagers bruk, og gjor at betennelsesreaksjonen i lungene til barnet demper seg. Dette hindrer barnets astma/betennelsesreaksjon i a utvikle seg og hindrer utvikling av sykdommen.",
		effect: "Alle sprayinhalasjoner ma gis pa inhalasjonskammer (Aerochamber, Optichamber, Babyhaler eller lignende) for a sikre at barnet far pustet inn medisinen pa riktig mate. Etter inhalasjon med steroider ma barnet alltid skylle munn, drikke eller pusse tenner for a fjerne rester av pulver fra munnen. Blir restene igjen i munnen kan det oppsta en smertefull soppinfeksjon i munnen."
	}
}
\end{lstlisting}
			We see from the JSON data that the module works by checking the query 
			and the variable \code{sqlsuccess}. We get the information that the medicine 
			with $ID=1$ has three types of instructions: an image given by an URI, 
			a general information field and a field to explain the effects.\\
			\hline
		\end{tabular}
	\end{center}
	\caption{Unit test 5.8, \code{get\_instructions.php}}
	\label{tab:unit5.8}
\end{table}

\begin{table} %UNIT 5.9: get_log_days_for_child
	\begin{center}
		\begin{tabular}{|p{3.0cm}|p{14.1cm}|}
			\hline
			\bf{Item} & \bf{Description}\\
			\hline
			\bf{ID} & UNIT5.9\\
			\bf{Description} & Test of the web access module \code{get\_log\_days\_for\_child.php}.\\
			\bf{Date} & 03.11.12\\
			\bf{Responsible} & Yngve\\
			\bf{Subject} & The database and \code{get\_log\_days\_for\_child.php}\\
			\bf{Precondition} & A working database with the tables \code{DAY\_MEDICINE\_DOSES} and \code{CHILDREN\_LOG\_DAYS}.\\
			\bf{Steps} &
			\begin{tabulenum}
				\item Initialize a web browser with the GET url \url{http://folk.ntnu.no/yngvesva/blopp/get\_log\_days\_for\_child.php?child\_id=6}
				\item Observe the resulting JSON data.
			\end{tabulenum}\\
			\hline
			\bf{Results} & The returned JSON data was:
\begin{lstlisting}[caption=Returned JSON data for \code{get\_log\_days\_for\_child.php}]
{
    sqlsuccess: true,
    child_id: "6",
    year: "2012",
    month: "11",
    query: "SELECT * FROM `DAY_MEDICINE_DOSES` WHERE `child_id`='6' AND YEAR(`day_date`)='2012' AND MONTH(`day_date`)='11' LIMIT 0,100",
    statuses_query: "SELECT `date`, `child_id`, `health_state_id` FROM `CHILDREN_LOG_DAYS` WHERE `child_id`='6' ORDER BY `date` ASC",
    days: [{
        date: "2012-11-01",
        health_state_id: "1",
        doses: []
    }, {
        date: "2012-11-02",
        health_state_id: "1",
        doses: [{
            id: "126",
            reward: "1",
            time: "00:00:01",
            day_date: "2012-11-02",
            child_id: "6",
            medicine_id: "1",
            health_state_id: "1",
            medical_plan_dose_id: "0",
            pollen_state_id: "0"
        }]
    }, {
        date: "2012-11-03",
        health_state_id: "3",
        doses: []
    }]
}
\end{lstlisting}
			As we see from the returned data, the child with $ID=6$ has one registered dose 
			in november, and the date was the third. We also see that the health state was
			changed to $health\_state\_id=3$ on the third. It is clear that the module is
			working because of the returned data, validating the queries and the variable
			\code{sqlsuccess}. 
			\\
			\hline
		\end{tabular}
	\end{center}
	\caption{Unit test 5.9, \code{get\_log\_days\_for\_child.php}}
	\label{tab:unit5.9}
\end{table}

\begin{table} %UNIT 5.10: get_log_for_child
	\begin{center}
		\begin{tabular}{|p{3.0cm}|p{14.0cm}|}
			\hline
			\bf{Item} & \bf{Description}\\
			\hline
			\bf{ID} & UNIT5.10\\
			\bf{Description} & Test of the web access module \code{get\_log\_for\_child.php}.\\
			\bf{Date} & 03.11.12\\
			\bf{Responsible} & Yngve\\
			\bf{Subject} & The database and \code{get\_log\_for\_child.php}.\\
			\bf{Precondition} & A working database with the table \code{DAY\_MEDICINE\_DOSES}.\\
			\bf{Steps} &
			\begin{tabulenum}
				\item Initialize a web browser with the GET url \url{http://folk.ntnu.no/yngvesva/blopp/get\_log\_for\_child.php?child\_id=6}.
				\item Observe the returned JSON data.
			\end{tabulenum}\\
			\hline
			\bf{Results} & The resulting JSON data was:
\begin{lstlisting}[caption=Returned JSON data from \code{get\_log\_for\_child.php}]
{
    sqlsuccess: true,
    child_id: "6",
    year: "2012",
    month: "11",
    query: "SELECT * FROM `DAY_MEDICINE_DOSES` WHERE `child_id`=6 AND YEAR(`day_date`)=2012 AND MONTH(`day_date`)=11 LIMIT 0,100",
    days: [
        {
            id: "126",
            reward: "1",
            time: "00:00:01",
            day_date: "2012-11-02",
            child_id: "6",
            medicine_id: "1",
            health_state_id: "1",
            medical_plan_dose_id: "0",
            pollen_state_id: "0"
        }
    ]
}
\end{lstlisting}
			We see from the returned data that there was one logged medication
			and that the module works from checking the query and the variable
			\code{sqlsuccess}.
			\\
			\hline
		\end{tabular}
	\end{center}
	\caption{Unit test 5.10, \code{get\_log\_for\_child.php}}
	\label{tab:unit5.10}
\end{table}

\begin{table} %UNIT 5.11: get_plan
	\begin{center}
		\begin{tabular}{|p{3.0cm}|p{14.0cm}|}
			\hline
			\bf{Item} & \bf{Description}\\
			\hline
			\bf{ID} & UNIT5.11\\
			\bf{Description} & Test of the web access module \code{get\_plan.php}.\\
			\bf{Date} & 04.11.12\\
			\bf{Responsible} & Yngve\\
			\bf{Subject} & The database and \code{get\_plan.php}.\\
			\bf{Precondition} & A working database with the tables \code{MEDICAL\_PLAN\_DOSES} and \code{CHILDREN}.\\
			\bf{Steps} &
			\begin{tabulenum}
				\item Initialize a web browser with the GET url \url{http://folk.ntnu.no/yngvesva/blopp/get\_plan.php?child\_id=6}.
				\item Observe the returned JSON data.
			\end{tabulenum}\\
			\hline
			\bf{Results} & The returned JSON data was:
\begin{lstlisting}[caption=Returned JSON data for \code{get\_plan.php}]
{
	sqlsuccess: true,
	child_id: "6",
	query: "SELECT Mpd.id, Mpd.medical_plan_id, Mpd.health_state_id, Mpd.time, Mpd.medicine_id, M.color AS 'medicine_color', M.name AS 'medicine_name' FROM `MEDICAL_PLAN_DOSES` AS Mpd, `MEDICINES` AS M WHERE Mpd.medical_plan_id IN (SELECT medical_plan_id FROM `CHILDREN` C WHERE C.id=6) AND Mpd.medicine_id = M.id LIMIT 0,100",
	rows: [
		{
			id: "41",
			medical_plan_id: "5",
			health_state_id: "1",
			time: "01:09:00",
			medicine_id: "1",
			medicine_color: "BLUE",
			medicine_name: "Flutide"
		}
	]
}
\end{lstlisting}
			We see from the returned data that there is one medication of
			Flutide planned at 01:09:00. We conclude that the module is
			working from looking at the query and checking the variable
			\code{sqlsuccess}.
			\\
			\hline
		\end{tabular}
	\end{center}
	\caption{Unit test 5.11, $get\_plan.php$}
	\label{tab:unit5.11}
\end{table}

\begin{table} %UNIT 5.12: register_medicine_taken
	\begin{center}
		\begin{tabular}{|p{3.0cm}|p{14.0cm}|}
			\hline
			\bf{Item} & \bf{Description}\\
			\hline
			\bf{ID} & UNIT5.12\\
			\bf{Description} & Test of the web access module \code{register\_medicine\_taken.php}.\\
			\bf{Date} & 04.11.12\\
			\bf{Responsible} & Yngve\\
			\bf{Subject} & The database and \code{register\_medicine\_taken.php}.\\
			\bf{Precondition} & A working database with the tables \code{DAY\_MEDICINE\_DOSES}, \code{HEALTH\_STATES} and \code{CHILDREN}.\\
			\bf{Steps} &
			\begin{tabulenum}
				\item Initialize a REST client with the URL \url{http://folk.ntnu.no/yngvesva/blopp/register\_medicine\_taken.php}.
				\item Add the POST parameters:
					\begin{tabulitem}
					  \item \code{child\_id}: 6
					  \item \code{medicine\_id}: 1
					  \item \code{day\_date}: 2012-09-03
					  \item \code{health\_state\_id}: 2
					\end{tabulitem}
				\item Observe the returned JSON data.
			\end{tabulenum}\\
			\hline
			\bf{Results} & The returned JSON data was:
\begin{lstlisting}[caption=Returned JSON data from \code{register\_medicine\_taken.php}]
{
    "sqlsuccess": true,
    "post": {
        "child_id": "6",
        "medicine_id": "1",
        "day_date": "2012-09-30",
        "health_state_id": "2"
    },
    "q": "INSERT INTO `DAY_MEDICINE_DOSES` (`id`, `reward`, `time`, `day_date`, `child_id`, `medicine_id`, `health_state_id`, `medical_plan_dose_id`, `pollen_state_id`) VALUES ('', '2', '00:00:01', '2012-09-30', '6', '1', '2', '', '')",
    "reward": "2",
    "unique": true
}
\end{lstlisting}
			We see from the returned data, by looking at \code{query} and \code{sqlsuccess},
			that an entry was added to the log. It was given a default time of
			00:00:01 since we did not specify it, and the returned calculated
			reward was 2. We can conclude that the dose was registered
			successfully and the module works.
			\\
			\hline
		\end{tabular}
	\end{center}
	\caption{Unit test 5.12, \code{register\_medicine\_taken.php}}
	\label{tab:unit5.12}
\end{table}

\begin{table} %UNIT 5.13: remove\_plan\_dose.php
	\begin{center}
		\begin{tabular}{|p{3.0cm}|p{14.0cm}|}
			\hline
			\bf{Item} & \bf{Description}\\
			\hline
			\bf{ID} & UNIT5.13\\
			\bf{Description} & Test of the web access module \code{remove\_plan\_dose.php}.\\
			\bf{Date} & 04.11.12\\
			\bf{Responsible} & Yngve\\
			\bf{Subject} & The database and \code{remove\_plan\_dose.php}.\\
			\bf{Precondition} & A working database with the table \code{MEDICAL\_PLAN\_DOSES}.\\
			\bf{Steps} &
			\begin{tabulenum}
				\item Initialize a REST client (POSTMAN) with the URL \url{http://folk.ntnu.no/yngvesva/blopp/remove\_plan\_dose.php}.
				\item Add the POST parameter:
					\begin{tabulitem}
					  \item $id$: 41
					\end{tabulitem}
				\item Observe the returned JSON data.
			\end{tabulenum}\\
			\hline
			\bf{Results} & The returned JSON data was:
\begin{lstlisting}[caption=Returned JSON data from \code{remove\_plan\_dose.php}]
{
    "sqlsuccess": true,
    "q": "DELETE FROM `MEDICAL_PLAN_DOSES` WHERE id='41'",
    "id": "41",
    "num_deleted": 1
}
\end{lstlisting}
			We conclude from the query, \code{sqlsuccess} and \code{num\_deleted} that
			the row with $ID=41$ was deleted and that the module works.
			\\
			\hline
		\end{tabular}
	\end{center}
	\caption{Unit test 5.13, \code{remove\_plan\_dose.php}}
	\label{tab:unit5.13}
\end{table}

\begin{table} %UNIT 5.14: remove_plan_medicine_at_time
	\begin{center}
		\begin{tabular}{|p{3.0cm}|p{14.0cm}|}
			\hline
			\bf{Item} & \bf{Description}\\
			\hline
			\bf{ID} & UNIT5.14\\
			\bf{Description} & Test of the web access module \code{remove\_plan\_medicine\_at\_time.php}.\\
			\bf{Date} & 04.11.12\\
			\bf{Responsible} & Yngve\\
			\bf{Subject} & The database and \code{remove\_plan\_medicine\_at\_time.php}\\
			\bf{Precondition} & A working database \\
			\bf{Steps} &
			\begin{tabulenum}
				\item Initiate a REST client (POSTMAN) with the URL \url{http://folk.ntnu.no/yngvesva/blopp/remove\_plan\_medicine\_at\_time.php}.
				\item Add the POST parameters:
					\begin{tabulitem}
					  \item \code{time}: 12:34:56
					  \item \code{medicine\_id}: 1
					  \item \code{child\_id}: 6
					  \item \code{health\_state\_id}: 3
					\end{tabulitem}
				\item Observe the returned JSON data.
			\end{tabulenum}\\
			\hline
			\bf{Results} & The returned data is:
\begin{lstlisting}[caption=Returned JSON data from \code{remove\_plan\_medicine\_at\_time.php}]
{
    "sqlsuccess": true,
    "q": "INSERT INTO `MEDICAL_PLAN_DOSES` (`id`, `medical_plan_id`, `health_state_id`, `time`, `medicine_id`) SELECT '', C.medical_plan_id, '3', '12:34:56', '1' FROM `CHILDREN` AS C WHERE C.id='6'",
    "child_id": "6",
    "health_state_id": "3",
    "medicine_id": "1",
    "time": "12:34:56",
    "id": 44
}
\end{lstlisting}
			We see from the data that a medication dose was removed, and
			by the query and \code{sqlsuccess} we see that the module works.
			\\
			\hline
		\end{tabular}
	\end{center}
	\caption{Unit test 5.14, \code{remove\_plan\_medicine\_at\_time.php}}
	\label{tab:unit5.14}
\end{table}

\begin{table} %UNIT 5.15, set_child_state
	\begin{center}
		\begin{tabular}{|p{3.0cm}|p{14.0cm}|}
			\hline
			\bf{Item} & \bf{Description}\\
			\hline
			\bf{ID} & UNIT5.15\\
			\bf{Description} & Test of the web access module \code{set\_child\_state.php}.\\
			\bf{Date} & 05.11.12\\
			\bf{Responsible} & Yngve\\
			\bf{Subject} & The database and \code{set\_child\_state.php}.\\
			\bf{Precondition} & A working database with the tables \code{CHILD\_HEALTH\_STATES} and \code{CHILDREN\_LOG\_DAYS}.\\
			\bf{Steps} &
			\begin{tabulenum}
				\item Initialize a REST client (POSTMAN) with the URL \url{http://folk.ntnu.no/yngvesva/bopp/set\_child\_state.php}.
				\item Add the POST parameters:
					\begin{tabulitem}
					  \item \code{child\_id}: 6
					  \item \code{state\_id}: 2
					\end{tabulitem}
				\item Observe the returned JSON data.
			\end{tabulenum}\\
			\hline
			\bf{Results} & The returned JSON data was:
\begin{lstlisting}[caption=Returned JSON from \code{set\_child\_state.php}]
{
    "sqlsuccess": true,
    "child_id": "6",
    "state_id": "2",
    "update_query": "UPDATE `CHILD_HEALTH_STATES` SET applies_now = IF(health_state_id = '2', 1, 0) WHERE child_id='6'",
    "insert_query": "REPLACE INTO `CHILDREN_LOG_DAYS` (`date`, `child_id`, `pollen_state_id`, `health_state_id`)  VALUES ('2012-11-05', '6', '1', '2')"
}
\end{lstlisting}
			We see from the returned JSON data that two tables were updated: the child's current
			state, and the log of state changes. We conclude that the module works because of
			the queries and the variable \code{sqlsuccess}.
			\\
			\hline
		\end{tabular}
	\end{center}
	\caption{Unit test 5.15, \code{set\_child\_state.php}}
	\label{tab:unit5.15}
\end{table}

%\begin{table}
%	\begin{center}
%		\begin{tabular}{|p{3.0cm}|p{14.0cm}|}
%			\hline
%			\bf{Item} & \bf{Description}\\
%			\hline
%			\bf{ID} & \\
%			\bf{Description} & \\
%			\bf{Date} & \\
%			\bf{Responsible} & \\
%			\bf{Subject} & \\
%			\bf{Precondition} & \\
%			\bf{Steps} &
%			\begin{tabulenum}
%				\item ~
%			\end{tabulenum}\\
%			\hline
%			\bf{Results} & \\
%			\hline
%		\end{tabular}
%	\end{center}
%	\caption{Unit test X.Y, Z}
%	\label{tab:unitX.Y}
%\end{table}