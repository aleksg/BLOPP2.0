\section{Architectural Description}

This project is a prototype to test a concept based
on gamification on asthma-treatments. Thus, the main architectural qualities 
identified for the applications is modifiability and usability. 
The reasoning behind this decision is that the software should be easy to modify with extended 
functionality if the concept is proven successful. 
It should
also be intuitive to use, since many of our potential users are children at ages below eight years old. Usability is our top priority.


We chose Model-View-Controller (MVC) as the architectural pattern for GAPP and CAPP, as MVC seperates the
underlying data models from views, and is proven to enhance modifibility. 
It is also an integrated part in Android, through a class called \code{Activity}.
We will explain more about activities in Section \ref{sec:gapp-dev-view}.
  

