\subsection{Database Access Layer}
\label{sec:databaseAccessLayer}
The database access layer consists of 14 PHP files hosted at			%TODO: update this number ? 
\url{http://folk.ntnu.no/yngvesva/blopp/}:
\begin{description}
    \item[add\_child.php] Takes a name, personal number (SSN) and a list of states (integer IDs) that the child can have. 
    	Creates an entry in the \code{CHILDREN} table for the child, and a medical plan entry. Also creates an entry 
    	in \code{CHILD\_HEALTH\_STATES} for all the states the child can have. Returns the generated \code{medical\_plan\_id}.
    \item[add\_plan\_dose.php] Inserts a new entry in the \code{MEDICAL\_PLAN\_DOSES} table for the given child, with the parameters 'health state', 'dose of the given medicine' and a timestamp. This is the primary module used to alter medical plans. Returns the ID of the
  		added dose.
    \item[dose\_is\_taken.php] Check if a dose of a planned medicine has been taken that day. Takes an \code{id} of an entry in
    	the table \code{MEDICAL\_PLAN\_DOSES} as input.
    \item[get\_available\_child\_states.php] Takes a \code{child\_id} and returns a list of the labels (colors, names) and IDs of the 
    	states the child can have.
    \item[get\_child.php] Takes an ID of a child and returns all the columns for the given ID in the \code{CHILDREN} table.
    \item[get\_child\_state.php] Accepts a child ID and returns the ID and label of the current state of the child.
    \item[get\_doses\_for\_current\_state.php] Takes a child ID and returns a list of planned doses of medicines that are not 
    	taken that day. The fields of each entry are: \code{id}, \code{medical\_plan\_id}, \code{health\_state\_id}, \code{time}, 
    	\code{medicine\_id}, \code{medicine\_karotz\_color} and \code{medicine\_name}.
    \item[get\_instructions.php] Get instructions (image, effect description and usage description) for a given medicine by ID.
    \item[get\_log\_days\_for\_child.php] For the calendar in GAPP, it was advantageous to have a database access method that
  		could return a list of days in a month with the child health state for each day, and a list of doses taken on that day. 
  		\code{get\_log\_days\_for\_child.php} accomplishes this by using the table \code{CHILDREN\_LOG\_DAYS} to find the latest recorded
  		health state before the given month started. Then it iterates through all the days, checking if there are any days in the month
  		where the status changes, and adding all doses, taken from \code{DAY\_MEDICINE\_DOSES}, on that day. The method takes a \code{child id},
  		and two optional parameters \code{month} and \code{year}. If the month and year are not set, the values for the current days
  		are used.
    \item[get\_log\_for\_child.php] Returns all registered entries in the table \code{DAY\_MEDICINE\_DOSES} for the day for a given child (id) 
    	during the given month during the given year.
    \item[register\_medicine\_taken.php] Register a dose of medicine taken. Accepts a post object with the fields \code{child\_id}, \code{medicine\_id}, 
    	\code{time}, \code{day\_date}, \code{health\_state\_id} and \code{medical\_plan\_dose\_id}. If there is an entry for that dose id that day, the method does 
    	nothing and simply returns $unique=false$. Otherwise, it calculates a reward, and updates \code{DAY\_MEDICINE\_DOSES} with the entry. 
    	Returns the reward for that dose. Then it adds the calculated reward to the child's total credits in the \code{CHILDREN} table. If 
  		no time is given, a default time of 00:00:01 is set.
    \item[remove\_plan\_dose.php] Deletes the entry in the table \code{MEDICAL\_PLAN\_DOSES} which corresponds to a given id. Returns
  		the number of deleted rows.
    \item[remove\_plan\_medicine\_at\_time.php] During development of GAPP, it was discovered that using a
  		combination of \code{child\_id}, \code{medicine\_id} and \code{time} as the key for removing elements in the \code{MEDICAL\_PLAN\_DOSES}
  		table would be easier than using an ID. Therefore the need for this module arised, and it simply removes all entries in the
  		table that fit the criteria.
    \item[set\_child\_state.php] Takes a child ID and a state ID and sets the current state of the given child to the specific state given as input. Also updates 
    	the table \code{CHILDREN\_LOG\_DAYS} with the given health state.
\end{description}